\documentclass[letterpaper]{article}
\usepackage[margin=1in]{geometry}
\usepackage{amsmath}
\usepackage{dsfont}
\usepackage{graphicx}
\usepackage[utf8]{inputenc}
\usepackage{xcolor}
% \usepackage[legalpaper, margin=2in]{geometry}
\parindent0pt
\parskip 1.5ex plus 1ex minus .5ex

\DeclareMathOperator{\Tr}{Tr}
\newcommand{\Ell}{\mathcal{L}}
\newcommand{\R}{\mathds{R}}
\newcommand{\C}{\mathds{C}}


\title{Open System Quantum Control}
% \author{Stefanie G{\"u}nther}
% \affil{Lawrence Livermore National Laboratory, CA, USA}
% \date{}

\begin{document}
\maketitle


We model open quantum systems with $Q$ oscillators with $n_k$ levels for the $k$-th oscillator, $k=1,\dots,Q$. We solve the Lindblad master equation
\begin{align}\label{mastereq}
  \dot \rho(t) = &-i(H(t)\rho(t) - \rho(t)H(t)) \\
          &+ \sum_{l=1}^2 \sum_{k=1}^Q \gamma_{lk} \left( \Ell_{lk} \rho(t) \Ell_{lk}^{\dagger} - \frac 1 2 \left( \Ell_{lk}^{\dagger}\Ell_{lk} \rho(t) + \rho(t)\Ell_{lk}^{\dagger} \Ell_{lk}\right) \right)
\end{align}
for the density matrix $\rho(t)\in \C^{N\times N}$, $N := \prod_{k=1}^N n_k$. The Hamiltonian $H(t)$ consists of a constant \textit{drift} part, and a time-varying \textit{control} part. For the \textit{rotating frame}, these are computed from
\begin{align}
  H(t) &= H_d + H_c(t) \\
  \text{with} \quad H_d &:= \sum_{k=1}^Q \left(- \frac{\xi_k}{2} a_k^{\dagger}a_k^{\dagger}a_k a_k - \sum_{l\neq k} \xi_{lk} a_l^{\dagger}a_l a_k^{\dagger} a_k  \right) \\
                 H_c(t) &:= \sum_{k=1}^Q \left( p(\vec{\alpha}_k,t) (a_k + a_k^{\dagger}) + i p(\vec{\beta}_k,t)(a_k - a_k^{\dagger})  \right)
\end{align}
Here, $a_k$ denotes the lowering operator:
\begin{align}
  a_1 &:= a^{(n_1)} \otimes I_{n_2} \otimes \dots \otimes I_{n_Q}\\
  a_2 &:= I_{n_1} \otimes a^{(n_2)} \otimes \dots \otimes I_{n_Q}\\
  &\vdots \\
  a_Q &:= I_{n_1} \otimes I_{n_2} \otimes \dots \otimes a^{(n_Q)}\\
\end{align}
with 
\begin{align}
 a^{(n_k)} = \begin{pmatrix}
   0 & 1 &          &         &    \\
     & 0 & \sqrt{2} &         &     \\
     &   & \ddots   & \ddots  &    \\
     &   &          &         & \sqrt{n_k-1}  \\
     &   &          &         & 0   
 \end{pmatrix} \in \R^{n_k \times n_k}
\end{align}
The time-dependent control functions $p(\vec{\alpha}_k,t), p(\vec{\beta}_k,t)$ are real-valued pulses represented by a linear combination of B-splines: 
\begin{align*}
  p(\vec{x},t) = \sum_{l=1}^L x_l B_l(t) \quad \text{for} \quad \vec x \in \R^L
\end{align*}
for $L$ B-Spline functions $B_l(t)$. The amplitudes $\vec{\alpha}_k, \vec{\beta}_k \in \R^{L}$ serve as control parameters (\textit{design} variables) that we modify in order to realize a desired system behavior, giving a total of $2QL$ design parameters:
\begin{align*}
  z &:= \left( \vec{\alpha}_1, \vec{\beta}_1, \dots, \vec{\alpha}_Q, \vec{\beta}_Q \right), \in \mathds{R}^{2LQ} \\
    &=\left(\alpha_1^1,\dots,\alpha_1^L,\beta_1^1, \dots, \beta_1^L, \dots, \alpha_Q^1,\dots,\alpha_Q^L,\beta_Q^1, \dots, \beta_Q^1 \right) \\
\end{align*}
The control functions are in the \textit{rotating frame}. To convert them back to the \textit{Lab frame} use
\begin{align}
  f_k(t) = 2p(\vec{\alpha}_k, t) cos(w_k t) - 2 p(\vec{\beta}_k,t) sin(w_k,t)
\end{align}

The collapse operators $\Ell_{lk}$ in the Lindblad terms of the master equation \eqref{mastereq} can be of the following type:
\begin{itemize}
  \item ``T1'' -- Decay: $\Ell_{1k} = a_k$
  \item ``T2'' -- Dephasing: $\Ell_{2k} = a_k^{\dagger}a_k$
\end{itemize}
for each oscillator $k$. The constants $\gamma_{lk}$ are the inverse half-live for the corresponding collapse process $l$: $\gamma_{lk} = {\frac{1}{T_l}}$ (in nanoseconds, ns). Typical T1 decay time is between $10-100$ microseconds (us). T2 dephasing time is typically about half of T1 decay time. Decay then behaves like $\exp(-t/{T_1})$. Dephasing behaves like $\exp(-\frac{1}{2{T_2}})$.

The quantum control problem aims to realize a certain target gate at final time $T$, represented by a unitary matrix $V\in \C^{N\times N}$. For closed quantum systems, we know that $\rho$ evolves in time with the unitary transformation matrix $\rho(t) = U(t)\rho(0) U(t)^{\dagger}$ (because $\rho(t) = \sum_k p_k \psi_k(t)\psi_k(t)^{\dagger}$, and $\psi_k(t)$ evolve according to the (linear) Schroedinger equation giving $\psi(t) = U(t) \psi(0)$). Hence, for closed systems we want to match $V$ with $U(T)$ so that $q(T) = V\rho(0)V^{\dagger}$ holds at final time. For open systems, $\rho(t)$ might not be in a pure state, and we might have $\rho(t) \neq U(t)\rho(0) U(t)^{\dagger}$. However, the optimal control problem still aims to find the amplitudes $z\in \R^{2QL}$ such that $\rho(T)$ is as close as possible to $V\rho(0)V^{\dagger}$, for any initial state $\rho(0)$. We measure the distance in the Frobenius norm:
\begin{align}\label{optimproblem_matrix}
 \| \rho(T) - V\rho(0)V^{\dagger} \|^2_F \rightarrow \min  \qquad \text{for all} \quad \rho(0)
\end{align} 
where the Frobenius norm is defined as $\|A\|^2_F = \Tr(A^{\dagger}A)$.

The Lindblad master equation \eqref{mastereq} is in matrix form, describing the evolution of the densisy matrix $\rho = (\rho_1, \dots, \rho_N) \in \C^{N\times N}$. In order to solve this numerically, we vectorize the equation to receive an ODE for $q(t) := \text{vec}(\rho(t)) \in \C^{N^2}$. Using the relations
\begin{align}
  \text{vec}(AB) &= (I_N\otimes A)\text{vec}(B) = (B^T\otimes I_N)\text{vec}(A) \\
  \text{vec}(ABC) &= (C^T\otimes A)\text{vec}(B)
\end{align}
for square matrices $A,B,C\in\C^{N\times N}$, we can derive the vectorized form of the Lindblad master equation:
\begin{align}\label{mastereq_vectorized}
  &\dot q(t) = M(t) q(t) \quad  \text{where} \\
  &M(t) := -i(I_N\otimes H - H^T \otimes I_N) + \sum_{l,k=1}^{2,Q} \gamma_{lk} \left( \Ell_{lk}\otimes \Ell_{lk} - \frac 1 2 \left( I_N\otimes \Ell^T_{lk}\Ell_{lk} + \Ell^T_{lk}\Ell_{lk} \otimes I_N \right) \right)
\end{align}
Since this is a linear ODE for $q(t)$, we can make the ansatz $q(T) = X(T) q(0)$. The solution operator $X(t) \in \C^{N^2\times N^2}$ satisfies $X(0) = I_{N^2}$ and $\dot X(t) = M(t) X(t)$. The vectorization of the target gate yields $\text{vec}(V\rho(0)V^{\dagger}) = \bar V\otimes V q(0)$. Hence, minimizing \eqref{optimproblem_matrix} for all initial conditions $\rho(0)$ relates to minimizing the difference between $X(T)$ and $\bar V\otimes V$. We therefore solve the following optimal control problem:
\begin{align}\label{optimproblem_final}
  \min_z \, J(z) + \frac{\gamma}{2}R(z) := \frac {1}{2N^2} \| &X(T) - \bar V \otimes V \|^2_F + \frac{\gamma}{2} \|z\|^2_2 \\
  \text{s.t.} \quad \dot X(t) &= M(t) X(t)  \\
                    X(0) &= I_{N^2}
\end{align}
where a Tikhonov regularizatioe term $R(z)$ has been added in order to well-condition the system. Using the definition $G := \bar V\otimes V$, the objective function $J(z)$ is decomposed into
\begin{align}
  J(z) &= \frac{1}{2N^2} \| X(T) - G\|^2_F \\
       &= \frac{1}{2N^2} \left( X(T)^{\dagger}X(T) + G^{\dagger}G - 2Re(X(T)^{\dagger}G) \right) \\
       &= \frac{1}{2N^2} \left( \sum_{i=1}^{N^2} x_i(T)^{\dagger}x_i(T) + g_i^{\dagger}g_i - 2Re(x_i(T)^{\dagger}g_i) \right) \\
       &= \frac{1}{2N^2} \left( \sum_{i=1}^{N^2} \| x_i(T) - g_i) \|^2_2 \right) \\
\end{align}
where $x_i$ and $g_i$ denote the $i-th$ column of $X(T)$ and $G=\bar V\otimes V$, respectively. Note, that $x_i(T)$ is the solution of the vectorized master equation \eqref{mastereq_vectorized} for the initial condition $x_i(0) = e_i$ being the $i$-th unit vector in $\R^{N^2}$. Hence, minimizing the normalized Frobenius norm equals to minimizing the average $L_2$ norm of the individual solution operator components. The objective function $J(z)$ therefore is the average squared error between the final state and the target gate (averaged over the initial conditions). 

Since $X(0) = I_{N^2}$ spans a basis for all initial conditions $q(0) \in C^{N^2}$, the optimizer finds optimal control functions that minimize the distance 
\begin{align}
       \frac{1}{2} \| q(T) - \bar V\otimes V q(0)) \|^2_2 \quad \text{for any} \quad q(0) \in C^{N^2}.
\end{align}
 
\textit{Fidelity:} If $\rho(T)$ would be unitary (as in closed systems), we would have $X(T) = \bar U(T) \otimes U(T)$ where $U(t)$ is the solution operator for the Schroedinger equation. In that case, one can show that 
\begin{align}
  \frac{1}{2N^2}\|X(T) - G\|_F^2 = 1 - \frac{1}{N^2}|\Tr(U(T)^{\dagger}V)|^2,
\end{align}
which we call the \textit{infidelity}, and the trace-term is the \textit{fidelity}. However, since we include the Lindblad terms, $\rho(T)$ is not unitary, and hence we don't have the decomposition using $U$. For defining a fidelity we consider $F(z) := 1 - J(z)$. Note, however, that if the optimized objective is in the order $O(\delta)$, then the error $\epsilon_i$ of a final state component $i$ from the gate is rather of the order $O(\sqrt{\delta})$, since $J$ represents (half of) the average squared error $J = \frac{1}{2N^2}\sum_i \epsilon_i^2$ and so $\epsilon_i^2 \sim O(\delta)$ on average.

% \textit{In order to check how well the final state matches the target gate, we further measure the ``fidelity'' of the system by looking at the diagonal components of the density matrix. Denote by $D(T) = (d_1(T), \dots d_N(T)) \in \C^{N\times N}$ the matrix whose $i$-th column contains the diagonal elements of the density matrix at time $T$ for the initial condition $\rho(0) = E_{ii}$ (zero-matrix with a one only at diagonal element $i,i$). We compute the fidelity from 
% \begin{align}
%   F &= \frac{1}{N} \left|\Tr(D(T)^{\dagger}V) \right|^2 \\
%     &= \frac{1}{N} \left| \sum_{i=1}^N d_i(T)^{\dagger}v_i \right|^2 \\
%     &= \frac{1}{N} \left( Re\left(\sum_{i=1}^N d_i(T)^{\dagger}v_i\right)^2 + Im\left( \sum_{i=1}^N d_i(T)^{\dagger}v_i \right)^2 \right)
% \end{align}
% \textcolor{red}{This doesn't make so much sense... Think it through!}
% }

Target gates that we consider include
    \begin{align}
      V_{X} := \begin{bmatrix} 0 & 1 \\ 1 & 0  \end{bmatrix} \quad
      V_{Y} := \begin{bmatrix} 0 & -i \\ i & 0 \end{bmatrix} \quad
      V_{Z} := \begin{bmatrix} 1 & 0 \\ 0 & -1 \end{bmatrix} \quad 
      V_{Hadamard} := \frac{1}{\sqrt{2}} \begin{bmatrix} 1 & 1 \\ 1 & -1 \end{bmatrix} \\
      V_{CNOT} := \begin{bmatrix} 1  & 0 & 0 & 0 \\ 
                                   0  & 1 & 0 & 0 \\ 
                                   0  & 0 & 0 & 1 \\ 
                                   0  & 0 & 1 & 0 \\ 
                    \end{bmatrix}\quad 
      V_{GroundState} := \begin{bmatrix} 1  &   &      \\ 
                                            & 0 &      \\ 
                                            &   &  \ddots 
                    \end{bmatrix}
    \end{align}

%The rotation matrix for 
%\begin{enumerate}
%    \item single qubit gates is 
%        \begin{align}
%           R(t) = \exp(iw_kTa_1^{\dagger}a_1) = \begin{bmatrix} 1 & 0 \\ 0 & e^{iw_1T} \end{bmatrix}
%        \end{align}
%    \item 2-qubit gates is 
%    \begin{align}
%        R(t) &= \exp(iw_kTa_1^{\dagger}a_1)\exp(iw_kTa_2^{\dagger}a_2) \\
%             &= \begin{bmatrix} 1 & 0 & 0 & 0\\
%                          0 & e^{iw_2T} & 0 & 0 \\
%                          0 & 0 & e^{iw_1T} & 0 \\
%                          0 & 0 & 0 & e^{i(w_1+w_2)T}  \\
%            \end{bmatrix}
%    \end{align}
%\end{enumerate}

\section{Implementation}
  \subsection{Vectorized, real-valued system setup}
   We solve the vectorized master equation \eqref{mastereq_vectorized} in real-valued variables with $q(t) = u(t) + iv(t)$, evolving the real-valued states $u(t), v(t)\in \R^{N^2}$ with
   \begin{align}
     \dot q(t) &= M(t) q(t) \\
   \Leftrightarrow \quad \begin{bmatrix} \dot u(t) \\ \dot v(t) \end{bmatrix} &= 
   \begin{bmatrix} A(t) & -B(t) \\ B(t) & A(t) \end{bmatrix} \begin{pmatrix} u(t) \\ v(t) \end{pmatrix} \label{realvaluedODE}
   \end{align}
   where $M(t) = A(t) + i B(t)$, $A(t), B(t)\in \R^{N^2\times N^2}$. To assemble $A = Re(M)$ and $B = Im(M)$ consider
   \begin{align}
     -i(I_N \otimes H - H^T \otimes I_N) &= -I_N \otimes \left(iH_d + iH_c(t)\right) + \left(iH_d + iH_c(t)\right)^T \otimes I_N \\
     \text{and} \quad iH_d + iH_c(t) &= i H_d + i\left( \sum_k p(\vec{\alpha}_k,t)(a_k + a_k^{\dagger}) + ip(\vec{\beta}_k,t)(a_k - a_k^{\dagger})\right) \\
                    &= - \sum_k p(\beta_k,t)(a_k - a_k^{\dagger}) + i\left( H_d + \sum_k p(\alpha_k,t)(a_k+a_k^{\dagger}) \right) 
   \end{align}
   Hence $A$ and $B$ consist of a constant part $A_d, B_d$ and a time-varying part $A_c(t), B_c(t)$ with the following:
   \begin{align}
     A_d &=  \sum_{l,k=1}^{2,Q}\gamma_{lk} \left( \Ell_{lk}\otimes\Ell_{lk} - \frac 1 2 \left(I_N \otimes \Ell_{lk}^T\Ell_{lk} + \Ell_{lk}^T\Ell_{lk}\otimes I_N\right) \right)\\
     B_d &= -I_N \otimes H_d + H_d^T \otimes I_N \\
     A_c(t) &= \sum_k p(\vec{\beta}_k,t) \underbrace{\left( I_N \otimes \left(a_k - a_k^{\dagger}\right) - \left(a_k - a_k^{\dagger}\right)^T\otimes I_N \right)}_{=:A_c^k} \\
     B_c(t) &= \sum_k p(\vec{\alpha}_k,t) \underbrace{\left( - I_N \otimes \left(a_k + a_k^{\dagger}\right) + \left(a_k + a_k^{\dagger}\right)^T\otimes I_N \right)}_{=:B_c^k} 
   \end{align}
   In the code, we initialize and store the constant matricees $A_d,B_d,A_c^k,  B_c^k$, and use them as building blocks to evaluate 
   \begin{align}
     A(t) &= Re(M(t)) = A_d + \sum_kp(\beta_k, t)A_c^k \\
     B(t) &= Im(M(t)) = B_d + \sum_k p(\alpha_k, t)B_c^k
   \end{align}
   at time $t$, and finally build $M(t)$ from \eqref{realvaluedODE}.
  \subsection{Timestepper}
    The forward time-evolution can use any of Petsc's time-stepping integration schemes. However the adjoint time-evolution does can not. For this we rely on a hand-written time-stepper. Hand-written time-steppers include Backward-Euler, and (preferred) the \textit{implicit midpoint rule (IMR)}, based on implicit Runge-Kutta scheme. We prefer IMR because it is a simplectic integrator. Because IMR is self-adjoint, the same IMR scheme is used to solve the adjoint equation. 

    In order to choose a time-step size $\Delta t$, we do an eigenvalue analysis of the constant drift Hamiltonian $H_d =  -2\pi \sum_{k=1}^Q \frac{x_k}{2} a_k^{\dagger}a_k^{\dagger}a_ka_k + \sum_{l\neq k} x_kl a_l^{\dagger}a_l a_k^{\dagger}a_k$:
       \begin{align*}  
         \dot u = -i H_d u \qquad \text{with} \quad H_d^{\dagger}  = H_d
       \end{align*} 
       There exists a transformation $Y$ s.t. 
       \begin{align*}
         Y^{\dagger}H_d Y + \Lambda \qquad  \text{where} \quad Y^{\dagger} = Y
       \end{align*}
       where $\Lambda$ is a diagonal matrix containing the eigenvalues of $H_d$. Transform $\tilde u = Y^{\dagger} u$, then the ODE transforms to 
       \begin{align*}
         \dot \tilde u = -i \Lambda \tilde u \quad \Rightarrow \dot \tilde u_i = -i\lambda_i \tilde u_i \quad \Rightarrow \tilde u_i = a \exp(-i\lambda_i t)
       \end{align*}
       Therefore, the period for each mode is $\tau_i = \frac{2\pi}{|\lambda_i|}$, hence the shortest period is $\tau_{min} = \frac{2\pi}{\max_i\{|\lambda_i|}\}$. If we want $p$ discrete time points per period, then $p\Delta t = \tau_{min}$, hence 
       \begin{align*}
         \Delta t = \frac{\tau_{min}}{p} = \frac{2\pi}{p\max_i\{|\lambda_i|\}}
       \end{align*}
       Usually, for a first order scheme we would use something like $p=20$, second order maybe $p=10$. Ander's used $p=80$ for the test case. 

       If we want to include the time-varying Hamiltonian part $H = H_d + H_c(t)$ in the analysis, then we could use the constraints on the control parameter amplitudes to remove the time-dependency using their larges value instead and so the same analysis again. However this doesn't ensure that we resolve the time-scale of the control functions. 

  \subsection{Optimization}
    We use the \textit{HiOp} optimization package for solving the vectorized optimal control problem \eqref{optimproblem_final}, which applies an interior point L-BFGS method. Each objective function evaluation requires to solve the initial value problem  
        \begin{align*}
          \dot x_i(t) &= M(t) x_i(t) \quad \forall \, t\in (0,T) \\
          x_i(0) &= e_i
        \end{align*}
        for all initial unit vectors $e_i \in \R^{N^2}$, followed by the evaluation of the objective function 
        \begin{align}
          J(z) = \frac{1}{2N^2} \left(\sum_{i=1}^{N^2} \|x_i(T) - g_i\|^2_2  \right) + \frac{\gamma}{2} \| z\|^2_2
        \end{align}

    We impose box constraints on the control spline parameters for each oscillator $k=1,\dots, Q$:
        \begin{align}
          | \alpha^l_k| \leq a_{max}^k \quad \text{and} \quad | \beta^l_k|  \leq a_{max}^k \quad \forall \, l=1,\dots, L 
        \end{align}


  \subsection{Parallelization}
    Four levels of parallelization are planned: 
      \begin{enumerate}
        \item Parallelization over the $N^2$ initial conditions
        \item Time-parallelization (XBraid)
        \item Spatial parallelism for parallel linear algebra (Petsc)
        \item Parallel optimization (HiOp)
      \end{enumerate}
      The global communicator (MPI\_COMM\_WORLD) is split into four sub-communicator, one for each of the above. Currently only the first two levels are implemented, while Petsc and HiOp run on communicators with size 1. The total number of mpi processes ($np_{total}$) is divided into cores for braid and for the initial conditions like so
         \begin{align*}
           np_{braid} * np_{init} = np_{total}.
         \end{align*}
      The user specifies the size of the communicator for distributing the initial conditions ($np_{init}$) in the config file, with the following requirements:
      \begin{itemize}
        \item $np_{init} \leq n_{init}$, where $n_{init}$ is the total number of initial conditions that are considered (can be $1$, $N$, or $N^2$). This requirement is handled by the code, limiting $np_{init}$ if necessary.
        \item $\frac{n_{init}}{np_{init}} \in \mathds{N}$, so that each processor group owns the same number of initial conditions.
        \item $\frac{np_{total}}{np_{init}} \in \mathds{N}$, so that each processor group has the same number of cores for braid, i.e. $np_{braid} = \frac{np_{total}}{np_{init}}$.
      \end{itemize}
  
\end{document}
