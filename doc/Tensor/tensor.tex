\documentclass[letterpaper]{article}
\usepackage[margin=1in]{geometry}
\usepackage{amsmath}
\usepackage{dsfont}
\usepackage{graphicx}
\usepackage[utf8]{inputenc}
\usepackage{xcolor}
% \usepackage[legalpaper, margin=2in]{geometry}
\parindent0pt
\parskip 1.5ex plus 1ex minus .5ex

\DeclareMathOperator{\Tr}{Tr}
\newcommand{\Ell}{\mathcal{L}}
\newcommand{\R}{\mathds{R}}
\newcommand{\C}{\mathds{C}}


\title{Tensor contractions for the Lindblad master equation}
\author{Stefanie G{\"u}nther}
% \date{\today}

\begin{document}
\maketitle

\section{Tensor contractions}
Consider a matrix-matrix multiplication as rank-2 tensor contractions. For a general $A\in \R^{N\times N}$ and $\rho\in \R^{N\times N}$ the product $A\rho$ corresponds to a contraction of the rows of $A$ with the columns of $\rho$:
\begin{align}
    A \rho = y \quad \Leftrightarrow \quad y_{a,b} = A_a^{i} \rho_{i,b} = \sum_i A_a^i \rho_{i,b} \quad \forall a,b = 0,\dots, N
\end{align}
where subscript and superscript indeces correspond to rows and columns, respectively, $A_m^l = A_{m,l} = A(m,l)$. 

Similarly, when applying $A$ to $\rho$ from the right, one gets 
\begin{align}
    \rho A = y \quad \Leftrightarrow \quad y_{a,b} = \rho_a^{i} A_{i,b} = \sum_i \rho_a^i A_{i,b} \quad \forall a,b = 0,\dots, n \\
   = \sum_i A_i^b \rho_{a,i} = \sum_i (A^T)_b^i \rho^T_{i,a} = A_i^b \rho_{a,i}
\end{align}
contracting the rows of $A$ with the columns of $\rho$. 

Now consider a kronecker product of operators $A\otimes B$ with dimensions $A\in\R^{n_a\times n_a}, B\in\R^{n_b\times n_b}$ multiplied to a matrix $\rho \in \R^{n_an_b\times n_bn_b}$ of matching dimensions. We can think of $\rho$ as a rank-4 tensor $\rho\in\R^{n_a\times n_b \times n_a \times n_b}$. The application of $A\otimes B$ to the tensor $\rho$ then corresponds to the tensor contraction 
\begin{align}
    z := A\otimes B \rho = A_{j_2}^{i_2}\otimes B_{j_1}^{i_1} \rho_{i_1, i_2, i_1^{\prime}, \i_2^{\prime}}
\end{align}
contracting first $B$ with $\rho$ on the first dimension $i_1$, then $A$ on the second dimension $i_2$:
\begin{align}
    y_{j_1, i_2, i_1^{\prime}, i_2^{\prime}} := B_{j_1}^{i_1} \rho_{i_1, i_2, i_1^{\prime}, i_2^{\prime}} = \sum_{i_1} B_{j_1}^{i_1} \rho_{i_1, i_2, i_1^{\prime}, i_2^{\prime}} \quad \forall j_1,i_2,i_1^\prime, i_2^\prime\\
    z_{j_1, j_2, i_1^{\prime}, i_2^{\prime}} := A_{j_2}^{i_2} y_{j_1, i_2, i_1^{\prime}, i_2^{\prime}} = \sum_{i_2} A_{j_2}^{i_2} y_{j_1, i_2, i_1^{\prime}, i_2^{\prime}} \quad \forall j_1,j_2,i_1^\prime, i_2^\prime
\end{align}


\section{Tensor contraction for Lindblad master equation}
For a quantum system with $Q$ oscillators in $H_{n_1} \otimes \dots H_{n_Q}$, with $n_k$ levels for the $k$-the oscillator and density matrix $\rho \in \C^{N\times N}$ with $N=\prod_{k=1}^{Q}$. Consider the density matrix being a tensor of rank $2Q$: 
\begin{align}
    \rho_{i_1,\dots, i_Q, i_1^{\prime}, \dots, i_Q^{\prime}} =: \rho_{i_k .., i_k^\prime..}\in\C^{n_1\times \dots \times n_Q \times n_1 \times \dots \times n_Q}
\end{align}
All operations that are applied to $\rho$ in the Lindblad equation are of the form $I\otimes \dots \otimes M^{n_k} \otimes \dots \otimes I$, or $I\otimes \dots \otimes M^{n_l} \otimes \dots \otimes I \otimes \dots \otimes M^{n_l} \otimes \dots \otimes I$, hence applying those operators to the tensor $\rho$ acts in at most two of the dimensions. Example for a 3-partite systems $H^{n_1} \otimes H^{n_2} \otimes H^{n_3}$:
\begin{align}
    I^{i_3}_{j_3}\otimes M^{i_2}_{j_2} \otimes I^{i_1}_{j_1} \rho_{i_1i_2i_3i_1^{\prime}i_2^\prime i_3^\prime} = M^{i_2}_{j_2} \rho_{i_1i_2i_3i_1^{\prime}i_2^\prime i_3^\prime} = \sum_{i_2} M(j_2, i_2) \rho(i_1,i_2,i_3,i_1^{\prime},i_2^\prime, i_3^\prime)  =: y_{i_1,j_2,i_3,i_1^{\prime},i_2^\prime,i_3^\prime}
\end{align}
and when applying from the right:
\begin{align}
    \rho_{i_1i_2i_3i_1^{\prime}i_2^\prime i_3^\prime}I^{i_3^{\prime}}_{j_3}\otimes M^{i_2^\prime}_{j_2} \otimes I^{i_1^\prime}_{j_1}  = M^{j_2}_{i_2^\prime} \rho_{i_1i_2i_3i_1^{\prime}i_2^\prime i_3^\prime} = \sum_{i_2^\prime} M(i_2^{\prime}, j_2) \rho(i_1,i_2,i_3,i_1^{\prime}i_2^\prime, i_3^\prime)  =: y_{i_1,i_2,i_3,i_1^{\prime}j_2, i_3^\prime}
\end{align}

\begin{enumerate}
    \item \textbf{Constant drift Hamiltonian} $H_d = -\sum_k \frac{\xi}{2} \left(N_k^2 - N_k\right) - \sum_{l<k} \xi_{lk} N_lN_k$, with
      \begin{align}
      N_k = I_{n_1}\otimes \dots \otimes N^{n_k} \otimes \dots \otimes I_{n_Q} \quad \text{with} \quad N^{n_k} := \begin{pmatrix} 0 & & \\ & 1& \\ & & \ddots \end{pmatrix}
      \end{align}
      Application to the tensor $\rho_{i_k .., i_k^\prime}$ is hence a sum of contractions to each of the $k$ dimensions, such as
      \begin{align}
        \left(N_k\right)_{j_k}^{i_k} \rho_{i_k .., i_k^\prime ..} = \sum_{i_k} \underbrace{N(j_k, i_k)}_{ = j_k\delta_{j_ki_k} } \rho(i_k.., i_k^\prime ..) = j_k \rho_{i_1\dots j_k \dots i_Q i_1^\prime \dots i_Q^\prime}
      \end{align}
      And similarly
      \begin{align}
        \left(N_k^2-N_k^2\right)_{j_k}^{i_k} \rho_{i_k .., i_k^\prime ..} = \left(j_k^2 - j_k\right) \rho_{i_1\dots j_k \dots i_Q i_1^\prime \dots i_Q^\prime}
      \end{align}

      The mixed term $N_lN_k$ expands to $N_k = I_{n_1}\otimes \dots \otimes N^{n_l} \otimes \dots \otimes N^{(n_k)} \otimes \dots \otimes I_{n_Q}$ and their tensor contraction results in products of the form 
      \begin{align}
        (N_l)^{i_l}_{j_l} \otimes (N_k)_{j_k}^{i_k} \rho_{i_k .., i_k^\prime ..} = j_lj_k\rho(..j_l .. j_k.., i_k^\prime ..)
      \end{align}

      Applying $H_d$ to the tensor $\rho$ from the left therefore gives  
      \begin{align}
        H_d \rho_{i_k .., i_k^\prime ..} = \underbrace{\left(-\sum_k \frac{\xi_k}{2}(i_k^2 - i_k) - \sum_{l<k} \xi_{lk} (i_l i_k) \right)}_{=:h_d(i_k..)} \rho_{i_k .., i_k^\prime..} \quad \forall i_k, i_k^\prime
      \end{align}
      Applying $H_d$ to $\rho$ from the right contracts on dimensions $i_k^\prime$:
      \begin{align}
        \rho_{i_k .., i_k^\prime ..} H_d  = h_d(i_k^\prime..) \rho_{i_k .., i_k^\prime..} \quad \forall i_k, i_k^\prime
        \end{align}

    \item \textbf{Time-dependent control Hamiltonian} $H_c(t) = \sum_k p^k(t) (a_k + a_k^\dag) + iq^k(a_k - a_k^\dag)$, with  
      \begin{align}
      a_k = I_{n_1}\otimes \dots \otimes a^{n_k} \otimes \dots \otimes I_{n_Q} \quad \text{with} \quad a^{n_k} := \begin{pmatrix} 0 & 1 & \\ & 0 & \sqrt{2} \\ & & & \ddots \end{pmatrix}
      \end{align}
      Each term contracts to the $k-$th dimension:
      \begin{align}
        (a_k\pm a_k^\dag) \rho_{i_k..i_k^\prime ..} = (a^{n_k} \pm a^{n_k})^{i_k}_{j_k} \rho_{i_k..i_k^\prime} &= \sum_{i_k} \underbrace{(a^{n_l} \pm a^{n_k})_{j_k}^{i_k}}_{= \sqrt{j_k+1} \delta_{i_k,j_k+1,i_{k}} \pm \sqrt{j_k} \delta_{i_k,j_k-1}} \rho_{i_k.. i_k^{\prime}} \\
        &= \sqrt{j_k+1} \rho(j_k+1.., i_k^\prime..) \pm \sqrt{j_k} \rho_(j_k-1, i_k^\prime)
      \end{align} 
      Applying $H_c(t)$ from the left then gives 
      \begin{align}
        H_c(t) \rho_{i_k.., i_k^\prime ..} = \sum_k p^k(t) \left(\sqrt{i_k+1} \rho_{i_k+1..,i_k^\prime..} + \sqrt{i_k} \rho_{i_k-1..,i_k^\prime..} \right) + q^k(t) \left(\sqrt{i_k+1} \rho_{i_k+1..,i_k^\prime..} - \sqrt{i_k} \rho_{i_k-1..,i_k^\prime..} \right)
      \end{align}
      Similar, applying $H_c(t)$ from the right to $\rho$ contracts on the dimensions $i_k^\prime$:
      \begin{align}
        \rho_{i_k.., i_k^\prime ..} H_c(t) = \sum_k p^k(t) \left(\sqrt{i_k^\prime+1} \rho_{i_k..,i_k^\prime+1..} + \sqrt{i_k^\prime} \rho_{i_k..,i_k^\prime-1..} \right) + q^k(t) \left(\sqrt{i_k^\prime+1} \rho_{i_k..,i_k^\prime+1..} - \sqrt{i_k^\prime} \rho_{i_k..,i_k^\prime-1..} \right)
      \end{align}

    \item \textbf{Lindblad terms} $\sum_k \sum_{l\in{1,2}} \gamma_{lk} \Ell_{lk} \rho \Ell_{lk}^\dag - \frac{1}{2} \left( \Ell_{lk}^\dag \Ell_{lk} \rho + \rho \Ell_{lk}^\dag \Ell_{lk}\right)$.
    
    The dephasing collapse operator $\Ell_{2k} = a_k^\dag a_k$ is diagonal. Contraction with $\rho_{i_k..i_k^\prime..}$ yields
    \begin{align}
        \sum_k \left( \gamma_{2k} i_ki_k^\prime - \frac 12 \left( i_k^2 + (i_k^\prime)^2\right) \right) \rho_{i_k..i_k^\prime..}
    \end{align}

    The decay collapse operator $\Ell_{1k} = a_k$ contains off-diagonals, and yields
    \begin{align}
        \sum_k \gamma_{1k} \sqrt{i_k+1}\sqrt{i_k^\prime} \rho_{i_k+1.., i_k-1..} - \frac 12 \left( i_k+ i_k^\prime \right)\rho_{i_k.., i_k^\prime..} 
    \end{align}
      
\end{enumerate}

\section{Implementation}

\begin{itemize}
    \item To apply the above operators efficiently, collect those terms that access the same index set, such as $(i_k..,i_k^\prime)$ vs. $(i_k+1, i_k^\prime)$, etc. Even though all the above operations will be applied for all $i_k.., i_k^\prime..$, these loops should be separated for each of those different index sets in order to ensure optimized memory loading and caching.
    \item  Iterate over the slowest moving index first. I.e. if the mapping from the vectorized $q = \vec(\rho)$ to $rho_(i_k..,i_k^\prime)$ is given by
    \begin{align}
        rho(i_1,\dots,i_Q,i_1^\prime,\dots,i_Q^\prime) = q[i_1 + i_2n_1 + i_3n_1n_2 + \dots]
    \end{align}
    then iterate over $i_Q^\prime$ as the most outer loop, then moving backwards with $i_1$ being the most inner, i.e. fastest moving, iteration index.
    \item Parallelization: The terms that contain off-diagonals will require communication of neighbouring processes to access an index $i_k+1$ from $i_k$ $\rightarrow$ ghost-layers?
\end{itemize}



\end{document}