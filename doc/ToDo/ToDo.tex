\documentclass{article}
\usepackage[margin=1in]{geometry}
\usepackage{amsmath}
\usepackage{dsfont}
\usepackage{graphicx}
\usepackage[utf8]{inputenc}
\parindent0pt
\parskip 1.5ex plus 1ex minus .5ex

\DeclareMathOperator{\Tr}{Tr}
\newcommand{\Ell}{\mathcal{L}}
\newcommand{\R}{\mathds{R}}
\newcommand{\C}{\mathds{C}}

\title{TODO}
% \author{Stefanie G{\"u}nther}
% \affil{Lawrence Livermore National Laboratory, CA, USA}
% \date{}

\begin{document}
\maketitle

\begin{itemize}
  \item Modify braid access level option so that it writes the state only after the optimization has finished. 
  \item For primal runs: Add option to read the initial condition from file (or config?), then run only this one condition forward. 
  \item \textbf{Parallelization:}
      \begin{itemize}
        \item Allow more flexibility for processor distribution between initial conditions and xbraid
        \item Parallelize HiOp: Distribute spline parameters.
        \item Parallelize linear algebra (Find out why serial run on multiple processors is faster than serial run on one processor. Seems like Petsc is still operating on multiple ranks?)
      \end{itemize}
  \item \textbf{Generalize gate interface} to more than 2 levels!
  \item \textbf{Bsplines with carrier waves}
  \item \textbf{Tensor formulation}
        \begin{itemize}
          \item Petsc's MatShell to implement the action of $A(t)$ on a vector ($\rightharpoondown$ compare FD-stencil)
          \item Taco?
          \item Preconditioning and solver for linear system $\rightarrow$ Bartlet - Steward for solving silvesters equation ?
        \end{itemize}
  \item \textbf{Braid convergence and scaling study} (Ben Southworth):
        \begin{itemize}
          \item Numerical convergence study for forward Braid using implicit vs. explicit time-stepping schemes ($\rightarrow$ Petsc's timestepper).
          \item Compute eigenvalues of system matrix $A(t)$ for theoretical analysis. 
          \item Compare 1st and 2nd order methods
          \item Testcase: 2 Oscillators, each 4 levels (more than two levels so that the drift Hamiltonian includes more terms). 
          \item Plot CPU time over error for various methods (serial vs parallel, both for different time stepping schemes)
          \item Create reference solution by using some high-order method, super small time-step, maybe variable time-steps. 
          \item Test all for various initial conditions (different unit vectors)
          \item Test both with and without the Lindblad terms
        \end{itemize}  
  \item Robust optimization, uncertainty quantification
  \item GPU implementation
\end{itemize}


\end{document}