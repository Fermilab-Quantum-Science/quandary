\documentclass{article}
\usepackage[margin=1in]{geometry}
\usepackage{amsmath}
\usepackage{dsfont}
\usepackage{graphicx}
\usepackage[utf8]{inputenc}
\parindent0pt
\parskip 1.5ex plus 1ex minus .5ex

\DeclareMathOperator{\Tr}{Tr}
\newcommand{\Ell}{\mathcal{L}}
\newcommand{\R}{\mathds{R}}
\newcommand{\C}{\mathds{C}}

\title{TODO}
% \author{Stefanie G{\"u}nther}
% \affil{Lawrence Livermore National Laboratory, CA, USA}
% \date{}

\begin{document}
\maketitle

\begin{itemize}
  \item Matrix-free solver generalization to more than two subsystems
  \item Parallel tensor contractions.
  \item For matrix-free tensor formulation, store $\sqrt{i_k}$ instead of recomputing them inside the loop. Or rather, allow for variable coefficients inside the lowering operators. 
  \item During groundstate opimization: discourage higher energy levels:
     \begin{align*}
        \frac 1 T \int_0^T \left( \rho_{kk}\right)^2 \, dt, \quad k > 0
     \end{align*}
  \item Look at eigenvalues of the Hessian of the objective function (SLEPc library?)
  \item Add decay for control functions over time: Constraint for the control function amplitudes:
      \begin{align*}
          \sum_{l=1}^L w_l\alpha_l \leq c_{max}
      \end{align*}
        where $w_l$ are monotonically increasing (e.g. exponentially). This will push later control functions to zero. 
  \item \textbf{Braid convergence and scaling study} (Ben Southworth):
        \begin{itemize}
          \item Numerical convergence study for forward Braid using implicit vs. explicit time-stepping schemes ($\rightarrow$ Petsc's timestepper).
          \item Compute eigenvalues of system matrix $A(t)$ for theoretical analysis. 
          \item Compare 1st and 2nd order methods
          \item Testcase: 2 Oscillators, each 4 levels (more than two levels so that the drift Hamiltonian includes more terms). 
          \item Plot CPU time over error for various methods (serial vs parallel, both for different time stepping schemes)
          \item Create reference solution by using some high-order method, super small time-step, maybe variable time-steps. 
          \item Test all for various initial conditions (different unit vectors)
          \item Test both with and without the Lindblad terms
        \end{itemize}  
  \item Robust optimization, uncertainty quantification
  \item GPU implementation
\end{itemize}


\end{document}
