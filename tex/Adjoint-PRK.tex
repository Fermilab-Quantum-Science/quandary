\documentclass[11pt]{article}
\usepackage{amsfonts,amsmath,amssymb,amsthm,graphicx}
\usepackage{multirow}
\usepackage{booktabs}
\usepackage[caption=false]{subfig} 
\usepackage{color}
\usepackage{ifthen}

\setlength{\topmargin}{0 mm}
\setlength{\oddsidemargin}{5 mm}
\setlength{\evensidemargin}{5 mm}
\setlength{\textwidth}{150 mm}
\setlength{\textheight}{210 mm}

\newtheorem{lemma}{Lemma}
\newtheorem{theorem}{Theorem}
\newtheorem{remark}{Remark}
\newcommand{\dzx}{D_0^x}
\newcommand{\dzxp}{D_{0p}^x}
\newcommand{\wdzx}{\widetilde{D_0^x}}
\newcommand{\wdzxp}{\widetilde{D_{0p}^x}}
\newcommand{\dpx}{D_+^x}
\newcommand{\dmx}{D_-^x}
\newcommand{\dzy}{D_0^y}
\newcommand{\dzyp}{D_{0p}^y}
\newcommand{\wdzy}{\widetilde{D_0^y}}
\newcommand{\dpy}{D_+^y}
\newcommand{\dmy}{D_-^y}
\newcommand{\dzz}{D_0^z}
\newcommand{\wdzz}{\widetilde{D_0^z}}
\newcommand{\dpz}{D_+^z}
\newcommand{\dmz}{D_-^z}
\newcommand{\dzt}{D_0^t}
\newcommand{\dpt}{D_+^t}
\newcommand{\dmt}{D_-^t}
\newcommand{\ehx}{E_{1/2}^x}
\newcommand{\ehy}{E_{1/2}^y}
\newcommand{\ehz}{E_{1/2}^z}
%
\newcommand{\calo}{{\cal O}}
%
\newcommand{\ab}{{\mathbf a}}
\newcommand{\bb}{{\mathbf b}}
\newcommand{\db}{{\mathbf d}}
\newcommand{\eb}{{\mathbf e}}
\newcommand{\fb}{{\mathbf f}}
\newcommand{\gb}{{\mathbf g}}
\newcommand{\ib}{{\mathbf i}}
\newcommand{\jb}{{\mathbf j}}
\newcommand{\nb}{{\mathbf n}}
\newcommand{\pb}{{\mathbf p}}
\newcommand{\tb}{{\mathbf t}}
\newcommand{\rb}{{\mathbf r}}
\newcommand{\yb}{{\mathbf y}}
\newcommand{\zb}{{\mathbf z}}
\newcommand{\qb}{{\mathbf q}}
\newcommand{\ub}{{\mathbf u}}
\newcommand{\vb}{{\mathbf v}}
\newcommand{\wb}{{\mathbf w}}

\newcommand{\Ab}{{\mathbf A}}
\newcommand{\Bb}{{\mathbf B}}
\newcommand{\Eb}{{\mathbf E}}
\newcommand{\Fb}{{\mathbf F}}
\newcommand{\Ib}{{\mathbf I}}
\newcommand{\Hb}{{\mathbf H}}
\newcommand{\Kb}{{\mathbf K}}
\newcommand{\Lb}{{\mathbf L}}
\newcommand{\Pb}{{\mathbf P}}
\newcommand{\Qb}{{\mathbf Q}}
\newcommand{\Rb}{{\mathbf R}}
\newcommand{\Ub}{{\mathbf U}}
\newcommand{\Tb}{{\mathbf T}}
\newcommand{\Xb}{{\mathbf X}}

\newcommand{\Abb}{\mathbb{A}}
\newcommand{\Bbbb}{\mathbb{B}}
\newcommand{\Ebb}{\mathbb{E}}
\newcommand{\Fbb}{\mathbb{F}}
\newcommand{\Ibb}{\mathbb{I}}
\newcommand{\Hbb}{\mathbb{H}}
\newcommand{\Kbb}{\mathbb{K}}
\newcommand{\Lbb}{\mathbb{L}}
\newcommand{\Pbb}{\mathbb{P}}
\newcommand{\Qbb}{\mathbb{Q}}
\newcommand{\Rbb}{\mathbb{R}}
\newcommand{\Ubb}{\mathbb{U}}
\newcommand{\Tbb}{\mathbb{T}}
\newcommand{\Xbb}{\mathbb{X}}
\newcommand{\Ybb}{\mathbb{Y}}
\newcommand{\Zbb}{\mathbb{Z}}

\newcommand{\uh}{\hat{u}}
\newcommand{\vh}{\hat{v}}
\newcommand{\ph}{\hat{p}}
\newcommand{\qh}{\hat{q}}

\newcommand{\re}{{\rm Re}\,}
\newcommand{\im}{{\rm Im}\,}

\renewcommand{\arraystretch}{1.3}
%
\newcommand{\p}{\partial}
%
\newcommand{\eq}{\!\!\! = \!\!\!}
\newcommand{\om}{\omega}
%\newcommand{\divergence}{\nabla\cdot}
%\newcommand{\curl}{\nabla\times}

\newcommand{\alphab}{\boldsymbol{\alpha}}
\newcommand{\lambdab}{\boldsymbol{\lambda}}
\newcommand{\psib}{\boldsymbol{\psi}}
\newcommand{\phib}{\boldsymbol{\phi}}
\newcommand{\Psib}{\boldsymbol{\Psi}}

\newcommand{\rhob}{\boldsymbol{\rho}}
\newcommand{\kab}{\boldsymbol{\kappa}}
\newcommand{\etab}{\boldsymbol{\eta}}
\newcommand{\zetab}{\boldsymbol{\zeta}}
\newcommand{\sigmab}{\boldsymbol{\sigma}}
\newcommand{\omegab}{\boldsymbol{\omega}}
\newcommand{\Gb}{{\mathbf G}}
\newcommand{\kb}{{\mathbf k}}
\newcommand{\sbold}{{\mathbf s}}
\newcommand{\ba}{\begin{array}}
\newcommand{\ea}{\end{array}}
\newcommand{\be}{\begin{equation}}
\newcommand{\ee}{\end{equation}}
\newcommand{\bd}{\begin{displaymath}}
\newcommand{\ed}{\end{displaymath}}
\newcommand{\pa}{\partial}
\newcommand{\f}{\frac}
\newcommand{\drp}{D^r_+}
\newcommand{\drm}{D^r_-}
\newcommand{\dqp}{D^q_+}
\newcommand{\dqm}{D^q_-}
\newcommand{\dtqn}{\widetilde{{D^q_0}} }
\newcommand{\dtrn}{\widetilde{{D^r_0}} }
\newcommand{\dqn}{D^q_0}
\newcommand{\drn}{D^r_0}
\newcommand{\erh}{E^r_{1/2}}
\newcommand{\eqh}{E^q_{1/2}}

\def\dpl{D_+}
\def\dmi{D_-}

\newcommand{\ubbar}{\bar{\mathbf{u}}}
\newcommand{\ubar}{\bar{u}}

% Numerical solutions
% \newcommand{\vb}{\mathbf{v}}


% Grids
\newcommand{\xb}{\mathbf{x}}
\newcommand{\ybh}{\hat{\mathbf{x}}}
\newcommand{\xbh}{\hat{\mathbf{x}}}
\newcommand{\Ja}{J_{\alpha}}
\newcommand{\ga}{g_{\alpha}}
\newcommand{\Ma}{M_{\alpha}}

% Interpolation
\newcommand{\Nxy}{N_{\mathbf{x}\rightarrow\hat{\mathbf{x}}}}
\newcommand{\Nyx}{N_{\hat{\mathbf{x}}\rightarrow\mathbf{x}}}
\newcommand{\Nuv}{N_{\bar{\ub}_1\rightarrow\bar{\ub}_2}}
\newcommand{\Nvu}{N_{\bar{\ub}_2\rightarrow\bar{\ub}_1}}
\newcommand{\Nij}{N_{\bar{\ub}_i\rightarrow\bar{\ub}_j}}
\newcommand{\Nji}{N_{\bar{\ub}_j\rightarrow\bar{\ub}_i}}
\newcommand{\Px}{P}
\newcommand{\Pxh}{\hat{P}}

% Domains
\newcommand{\domp}{\Omega_{p}}
\newcommand{\domu}{\Omega_{\ub}}
\newcommand{\domv}{\Omega_{\vb}}
\newcommand{\domui}{\Omega_{\bar{\mathbf{u}}_i}}
\newcommand{\domuj}{\Omega_{\bar{\mathbf{u}}_j}}
\newcommand{\gamja}{\Gamma_{j0}}
\newcommand{\gamjb}{\Gamma_{j1}}
\newcommand{\gamia}{\Gamma_{i0}}
\newcommand{\gamib}{\Gamma_{i1}}

% Comments
\newcommand{\red}{\color{red} AP:}
\newcommand{\usecomments}{true}
\newcommand{\ocomment}[1] {
\ifthenelse{ \equal{\usecomments}{true} }{
 \textbf{Ossian: }{\color{blue} #1}
}{
}
}



\begin{document}

%% \title{Notes on Schroedinger's equation}

%% \author{N. Anders Petersson\thanks{Center for Applied
%%     Scientific Computing, Lawrence Livermore National Laboratory, L-561, PO Box 808, Livermore CA
%%     94551. }}

%% \date{\today}

%% \maketitle
\section{The Schr\"odinger equation}
Consider the Schr\"odinger equation written as a system of ordinary differential equations (ODE),
\begin{equation}\label{eq_shrodinger}
\dot{\psi} = -i H(t) \psi,\quad t\geq 0,\quad \psi(0) = \psi_0,
\end{equation}
where $\psi(t)\in {\mathbb C}^N$ and $H\in {\mathbb C}^{N\times N}$ is a Hermitian matrix.

To utilize efficient numerical software it is desirable to derive a real-valued equivalent of
\eqref{eq_shrodinger}. Let the real-valued functions $u(t)$ and $v(t)$ hold the real and negative
imaginary parts of $\psi(t)$,
\[
\psi(t) = u(t) - iv(t)
\]
and decompose the total Hamiltonian matrix into $H(t) = K(t) + i S(t)$, where the real-valued
matrices $K$ and $S$ hold the symmeric and skew-symmetric parts of $H$: $K^T = K$ and $S^T =
-S$. We have,
\begin{align*}
H\psi &= (K+iS)(u - iv) = (Ku + Sv) + i(Su - Kv),\\
-iH\psi &= -i(Ku + Sv) + (Su - Kv).
\end{align*}
Therefore, a real-valued equivalent of the system \eqref{eq_shrodinger} is
\begin{equation}\label{eq_real-shrodinger}
  \begin{bmatrix} \dot{u}\\ \dot{v} \end{bmatrix} =
%
  \begin{bmatrix}
    S(t) & -K(t) \\ K(t) & S(t)
  \end{bmatrix}     
  %
  \begin{bmatrix} u\\ v \end{bmatrix} .
\end{equation}
This is a Hamiltonian system where the time-dependent Hamiltonian function is
\begin{equation}\label{eq_hamiltonian}
\kappa(u,v,t) = u^T S(t) v + \frac{1}{2} u^T K(t) u + \frac{1}{2} v^T K(t) v.
\end{equation}
In general, the Hamiltonian function is non-separable due to the term $u^TS(t) v$.

\subsection{Time integration}
We consider integrating \eqref{eq_real-shrodinger} with a time-dependent forcing,
\begin{equation}\label{eq_s-v-forcing}
  \begin{bmatrix} \dot{u}\\ \dot{v} \end{bmatrix} =
%
  \begin{bmatrix}
    S(t) & -K(t) \\ K(t) & S(t)
  \end{bmatrix}     
  %
  \begin{bmatrix} u\\ v \end{bmatrix}
  %
 % + \begin{bmatrix} F^u(t) \\ F^v(t) \end{bmatrix}
  %
  =:
%
  \begin{bmatrix}
    f(u,v,t)\\
    g(u,v,t)
  \end{bmatrix},\quad
  %
  \begin{bmatrix}
    u(0)\\
    v(0)
  \end{bmatrix}
  =
    \begin{bmatrix}
    u0\\
    v0
  \end{bmatrix},
%
\end{equation}
for $0 \leq t \leq T$. Here, $u\in \mathbb{R}^N$ and $v\in \mathbb{R}^N$; $K=K^T$ and $S=-S^T$ are real-valued $N\times N$
matrices.  To solve \eqref{eq_s-v-forcing} numerically, we discretize time on a grid with
$t_n = nh$, $n=0,1,2,\ldots$, where $h>0$ is the time step, which we assume to be constant (our approach
generalizes to variable time steps). We denote the numerical solution $u_n\approx u(t_n)$ and $v_n\approx v(t_n)$.  The
Stromer-Verlet scheme is a symplectic, time-reversible, second order accurate scheme that can be written as a
partitioned Runge-Kutta (PRK) scheme. It combines the trapezoidal and the implicit midpoint rules,
\begin{alignat}{3}
  u_0 &= u0,\quad & v_0 = v0,\\
  %
  u_{n+1} &= u_n + h \sum_{i=1}^2 b_i \kappa_i,\quad
  &v_{n+1} = v_n + h \sum_{i=1}^2 \hat{b}_i \ell_i,\label{eq_uv-update}\\
  %
  \kappa_i &= f(U_{n,i}, V_{n,i}, t_n + c_i h),\quad
  &\ell_i = g(U_{n,i}, V_{n,i}, t_n + \hat{c}_i h),\\
  %
  U_{n,i} &= u_n + h \sum_{j=1}^2 a_{ij} \kappa_j,\quad
  &V_{n,i} = v_n + h \sum_{j=1}^2 \hat{a}_{ij} \ell_j,\label{eq_stage-values}
\end{alignat}
for $i=1,2$. The coefficients for the trapezoidal rule are
\begin{align}
  a_{11} &= a_{12} = 0,\quad a_{21} = a_{22} = \frac{1}{2},\\
  b_{1} &= b_2 = \frac{1}{2},\quad  c_1 = 0,\quad c_2=1,
\end{align}
and for the implicit mid-point rule,
\begin{align}
  \hat{a}_{11} &= \hat{a}_{21} = \frac{1}{2},\quad \hat{a}_{12} = \hat{a}_{22} = 0,\\
  \hat{b}_{1} &= \hat{b}_2 = \frac{1}{2},\quad  \hat{c}_1 = \hat{c}_2=\frac{1}{2}.
\end{align}
Note that $b_i = \hat{b}_i$. In the following we will only use $b_i$.

For the system \eqref{eq_s-v-forcing}, the slopes are
\begin{align}
\kappa_1 &= S_nU_{n,1} - K_nV_{n,1} ,\label{eq_kappa1}\\
\kappa_2 &= S_{n+1}U_{n,2} - K_{n+1}V_{n,2},\\
\ell_1 &= K_{n+1/2} U_{n,1} + S_{n+1/2} V_{n,1},\\
\ell_2 &= K_{n+1/2} U_{n,2} + S_{n+1/2} V_{n,2},\label{eq_ell2}
\end{align}
where $S_n = S(t_{n})$, $S_{n+1/2} = S(t_n + 0.5 h)$, etc. We can express the PRK scheme in terms of
the stage values $U_{n,i}$ and $V_{n,i}$ by substituting \eqref{eq_kappa1}-\eqref{eq_ell2} into \eqref{eq_uv-update},
\begin{align}
  u_0 &= u0,\quad v_0 = v0,\label{eq_initial-cond}\\
  u_{n+1} &= u_n + \frac{h}{2}\left(S_nU_{n,1} - K_nV_{n,1} + S_{n+1}U_{n,2} - K_{n+1}V_{n,2}
  \right),\label{eq_u-update}\\
  %
  v_{n+1} &= v_n + \frac{h}{2}\left(K_{n+1/2} \left(U_{n,1} + U_{n,2}\right) + S_{n+1/2}\left(
  V_{n,1} +  V_{n,2}\right) \right),\label{eq_v-update}
\end{align}
and into \eqref{eq_stage-values},
\begin{align} 
  U_{n,1} &= u_n,\label{eq_U1}\\
  U_{n,2} &= u_n + \frac{h}{2}\left(S_nU_{n,1} - K_nV_{n,1} + S_{n+1}U_{n,2} -
  K_{n+1}V_{n,2} \right),\\
  V_{n,1} &= v_n + \frac{h}{2}\left(  K_{n+1/2} U_{n,1} + S_{n+1/2} V_{n,1} \right),\\
  V_{n,2} &= v_n + \frac{h}{2}\left(  K_{n+1/2} U_{n,1} + S_{n+1/2} V_{n,1} \right).\label{eq_V2}
\end{align}
Note that $V_{n,1} = V_{n,2}$, but $\ell_1 \ne \ell_2$. Because $S(t)\ne0$, the Hamiltonian system
is non-separable and the Str\"omer-Verlet scheme becomes partially implicit. We remark that the order
of accuracy can be raised by using a compositional (Suzuki-Trotter) technique where each time-step
is decomposed into several sub-steps. Fourth order accuracy can be obtained with three sub-steps and
sixth order accuracy with seven or nine sub-steps. Eight order requires at least fifteen sub-steps, see
Hairer et al.~\cite{HairerLubichWanner-06} for further details.

\section{Calculating the gradient of an objective function}

Consider the case where the matrices $K$ and $S$ in \eqref{eq_s-v-forcing} depend on a parameter
vector $\alpha \in {\mathbb R}^D$,
\[
K = K(t,\alpha),\quad S = S(t,\alpha).
\]
For a given parameter vector $\alpha$, we are interested in numerically evaluating the continuous objective function,
\[
{\cal J}(u(\alpha), v(\alpha)) = \int_{0}^T \left( \langle u(t),W u(t) \rangle +  \langle v(t),W v(t) \rangle\right)\, dt,
\]
where $W$ is a real $N\times N$ matrix and $(u,v)$ satisfies the ODE \eqref{eq_s-v-forcing}. Here, $(u,v)$ implicitly depend on $\alpha$
through the matrices $K$ and $S$.

We can calculate a numerical approximation of the solution to the ODE \eqref{eq_s-v-forcing} by
using the above Stromer-Verlet scheme. A second order accurate approximation of the continuous
objective function is given by
\begin{equation}\label{eq_discrete_obj}
  {\cal J}_h(U(\alpha), V(\alpha)) = h \sum_{n=0}^{N_t-1} \sum_{i=1}^2 b_i \left( \langle U_{n,1}, W U_{n,i}\rangle +
  %
  \langle V_{n,1}, W V_{n,i}\rangle \right),
\end{equation}
which we call the discrete objective function. Here, $(U_{n,i}, V_{n,i})$ are the stage values
\eqref{eq_U1}-\eqref{eq_V2} and $N_t$ is the number of time steps such that $T=h N_t$.
The discrete objective function is straight forward to evaluate during the time-stepping of the
Stromer-Verlet scheme.

\subsection{The Lagrangian of the discretized problem}
To minimize ${\cal J}_h$ using a gradient-based approach such as BFGS we need to evaluate the gradient
$d {\cal J}_h/ d\alpha_q$, for $q=1,2,\ldots,D$. To ensure proper convergence of BFGS it is important that the
gradient of ${\cal J}_h$ is calculated very accurately, i.e., without introducing additional truncation errors. We
adopt a discretize-before-optimize (DBO) approach based on minimizing a discrete Lagrangian. We start by
introducing the Lagrange multipliers $(\mu_n,\nu_n)$ and $(M_{n,j}, N_{n,j})$ and consider the discrete Lagrangian,
\begin{equation}
{\cal L}_h(u,v,U,V, \mu,\nu,M,N,\alpha) = {\cal J}_h(U,V) - \langle u_0 - u0, \mu_0\rangle - \langle
v_0 - v0, \nu_0\rangle - \sum_{k=1}^6 T_k,
\end{equation}
where,
\begin{align}
T_1 &= \sum_{n=0}^{N_T-1} \left\langle u_{n+1} - u_n - \frac{h}{2}\left(S_nU_{n,1} - K_nV_{n,1} + S_{n+1}U_{n,2} - K_{n+1}V_{n,2}
  \right), \mu_{n+1} \right\rangle,\\
T_2 &= \sum_{n=0}^{N_T-1} \left\langle v_{n+1} - v_n - \frac{h}{2}\left(K_{n+1/2} \left(U_{n,1} + U_{n,2}\right) + S_{n+1/2}\left(
  V_{n,1} +  V_{n,2}\right) \right), \nu_{n+1} \right\rangle,\\
T_3 &= \sum_{n=0}^{N_T-1} \left\langle U_{n,1} - u_n, M_{n,1} \right\rangle,\\
T_4 &= \sum_{n=0}^{N_T-1} \left\langle U_{n,2} - u_n - \frac{h}{2}\left(S_nU_{n,1} - K_nV_{n,1} + S_{n+1}U_{n,2} -
  K_{n+1}V_{n,2} \right), M_{n,2} \right\rangle,\\
T_5 &= \sum_{n=0}^{N_T-1} \left\langle V_{n,1} - v_n - \frac{h}{2}\left(  K_{n+1/2} U_{n,1} + S_{n+1/2} V_{n,1} \right), N_{n,1} \right\rangle,\\
T_6 &= \sum_{n=0}^{N_T-1} \left\langle V_{n,2} - v_n - \frac{h}{2}\left(  K_{n+1/2} U_{n,1} + S_{n+1/2} V_{n,1} \right), N_{n,2} \right\rangle.
\end{align}
The basic idea of the Lagrangian approach is to consider the variables $(u,v,U,V, \mu,\nu,M,N)$ as being independent of
$\alpha$; only the matrices $K_p$ and $S_q$ depend explicitly on $\alpha$. In the particular case
when $(u^*,v^*)$, with
corresponding stage values $(U^*,V^*)$, is a solution of the Stromer-Verlet scheme
\eqref{eq_initial-cond}-\eqref{eq_V2} for a given parameter vector $\alpha$,
\[
  {\cal L}_h(u^*,v^*,U^*,V^*, \mu,\nu,M,N,\alpha) = {\cal J}_h(U^*,V^*).
\]

\bibliographystyle{plain}
\bibliography{quantum}

\end{document}
