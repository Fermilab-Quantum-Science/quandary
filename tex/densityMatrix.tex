\documentclass[11pt]{article}
\usepackage{amsfonts,amsmath,amssymb,amsthm,graphicx}
\usepackage{multirow}
\usepackage{booktabs}
\usepackage[caption=false]{subfig} 
\usepackage{color}
\usepackage{ifthen}

\setlength{\topmargin}{0 mm}
\setlength{\oddsidemargin}{5 mm}
\setlength{\evensidemargin}{5 mm}
\setlength{\textwidth}{150 mm}
\setlength{\textheight}{210 mm}

\newtheorem{lemma}{Lemma}
\newtheorem{theorem}{Theorem}
\newtheorem{remark}{Remark}
\newcommand{\dzx}{D_0^x}
\newcommand{\dzxp}{D_{0p}^x}
\newcommand{\wdzx}{\widetilde{D_0^x}}
\newcommand{\wdzxp}{\widetilde{D_{0p}^x}}
\newcommand{\dpx}{D_+^x}
\newcommand{\dmx}{D_-^x}
\newcommand{\dzy}{D_0^y}
\newcommand{\dzyp}{D_{0p}^y}
\newcommand{\wdzy}{\widetilde{D_0^y}}
\newcommand{\dpy}{D_+^y}
\newcommand{\dmy}{D_-^y}
\newcommand{\dzz}{D_0^z}
\newcommand{\wdzz}{\widetilde{D_0^z}}
\newcommand{\dpz}{D_+^z}
\newcommand{\dmz}{D_-^z}
\newcommand{\dzt}{D_0^t}
\newcommand{\dpt}{D_+^t}
\newcommand{\dmt}{D_-^t}
\newcommand{\ehx}{E_{1/2}^x}
\newcommand{\ehy}{E_{1/2}^y}
\newcommand{\ehz}{E_{1/2}^z}
%
\newcommand{\calo}{{\cal O}}
%
\newcommand{\ab}{{\mathbf a}}
\newcommand{\bb}{{\mathbf b}}
\newcommand{\db}{{\mathbf d}}
\newcommand{\eb}{{\mathbf e}}
\newcommand{\fb}{{\mathbf f}}
\newcommand{\gb}{{\mathbf g}}
\newcommand{\ib}{{\mathbf i}}
\newcommand{\jb}{{\mathbf j}}
\newcommand{\nb}{{\mathbf n}}
\newcommand{\pb}{{\mathbf p}}
\newcommand{\tb}{{\mathbf t}}
\newcommand{\rb}{{\mathbf r}}
\newcommand{\yb}{{\mathbf y}}
\newcommand{\zb}{{\mathbf z}}
\newcommand{\qb}{{\mathbf q}}
\newcommand{\ub}{{\mathbf u}}
\newcommand{\vb}{{\mathbf v}}
\newcommand{\wb}{{\mathbf w}}

\newcommand{\Ab}{{\mathbf A}}
\newcommand{\Bb}{{\mathbf B}}
\newcommand{\Eb}{{\mathbf E}}
\newcommand{\Fb}{{\mathbf F}}
\newcommand{\Ib}{{\mathbf I}}
\newcommand{\Hb}{{\mathbf H}}
\newcommand{\Kb}{{\mathbf K}}
\newcommand{\Lb}{{\mathbf L}}
\newcommand{\Pb}{{\mathbf P}}
\newcommand{\Qb}{{\mathbf Q}}
\newcommand{\Rb}{{\mathbf R}}
\newcommand{\Ub}{{\mathbf U}}
\newcommand{\Tb}{{\mathbf T}}
\newcommand{\Xb}{{\mathbf X}}

\newcommand{\Abb}{\mathbb{A}}
\newcommand{\Bbbb}{\mathbb{B}}
\newcommand{\Ebb}{\mathbb{E}}
\newcommand{\Fbb}{\mathbb{F}}
\newcommand{\Ibb}{\mathbb{I}}
\newcommand{\Hbb}{\mathbb{H}}
\newcommand{\Kbb}{\mathbb{K}}
\newcommand{\Lbb}{\mathbb{L}}
\newcommand{\Pbb}{\mathbb{P}}
\newcommand{\Qbb}{\mathbb{Q}}
\newcommand{\Rbb}{\mathbb{R}}
\newcommand{\Ubb}{\mathbb{U}}
\newcommand{\Tbb}{\mathbb{T}}
\newcommand{\Xbb}{\mathbb{X}}
\newcommand{\Ybb}{\mathbb{Y}}
\newcommand{\Zbb}{\mathbb{Z}}

\newcommand{\uh}{\hat{u}}
\newcommand{\vh}{\hat{v}}
\newcommand{\ph}{\hat{p}}
\newcommand{\qh}{\hat{q}}

\newcommand{\re}{{\rm Re}\,}
\newcommand{\im}{{\rm Im}\,}

\renewcommand{\arraystretch}{1.3}
%
\newcommand{\p}{\partial}
%
\newcommand{\eq}{\!\!\! = \!\!\!}
\newcommand{\om}{\omega}
%\newcommand{\divergence}{\nabla\cdot}
%\newcommand{\curl}{\nabla\times}

\newcommand{\alphab}{\boldsymbol{\alpha}}
\newcommand{\lambdab}{\boldsymbol{\lambda}}
\newcommand{\psib}{\boldsymbol{\psi}}
\newcommand{\phib}{\boldsymbol{\phi}}
\newcommand{\Psib}{\boldsymbol{\Psi}}

\newcommand{\rhob}{\boldsymbol{\rho}}
\newcommand{\kab}{\boldsymbol{\kappa}}
\newcommand{\etab}{\boldsymbol{\eta}}
\newcommand{\zetab}{\boldsymbol{\zeta}}
\newcommand{\sigmab}{\boldsymbol{\sigma}}
\newcommand{\omegab}{\boldsymbol{\omega}}
\newcommand{\Gb}{{\mathbf G}}
\newcommand{\kb}{{\mathbf k}}
\newcommand{\sbold}{{\mathbf s}}
\newcommand{\ba}{\begin{array}}
\newcommand{\ea}{\end{array}}
\newcommand{\be}{\begin{equation}}
\newcommand{\ee}{\end{equation}}
\newcommand{\bd}{\begin{displaymath}}
\newcommand{\ed}{\end{displaymath}}
\newcommand{\pa}{\partial}
\newcommand{\f}{\frac}
\newcommand{\drp}{D^r_+}
\newcommand{\drm}{D^r_-}
\newcommand{\dqp}{D^q_+}
\newcommand{\dqm}{D^q_-}
\newcommand{\dtqn}{\widetilde{{D^q_0}} }
\newcommand{\dtrn}{\widetilde{{D^r_0}} }
\newcommand{\dqn}{D^q_0}
\newcommand{\drn}{D^r_0}
\newcommand{\erh}{E^r_{1/2}}
\newcommand{\eqh}{E^q_{1/2}}

\def\dpl{D_+}
\def\dmi{D_-}

\newcommand{\ubbar}{\bar{\mathbf{u}}}
\newcommand{\ubar}{\bar{u}}

% Numerical solutions
% \newcommand{\vb}{\mathbf{v}}


% Grids
\newcommand{\xb}{\mathbf{x}}
\newcommand{\ybh}{\hat{\mathbf{x}}}
\newcommand{\xbh}{\hat{\mathbf{x}}}
\newcommand{\Ja}{J_{\alpha}}
\newcommand{\ga}{g_{\alpha}}
\newcommand{\Ma}{M_{\alpha}}

% Interpolation
\newcommand{\Nxy}{N_{\mathbf{x}\rightarrow\hat{\mathbf{x}}}}
\newcommand{\Nyx}{N_{\hat{\mathbf{x}}\rightarrow\mathbf{x}}}
\newcommand{\Nuv}{N_{\bar{\ub}_1\rightarrow\bar{\ub}_2}}
\newcommand{\Nvu}{N_{\bar{\ub}_2\rightarrow\bar{\ub}_1}}
\newcommand{\Nij}{N_{\bar{\ub}_i\rightarrow\bar{\ub}_j}}
\newcommand{\Nji}{N_{\bar{\ub}_j\rightarrow\bar{\ub}_i}}
\newcommand{\Px}{P}
\newcommand{\Pxh}{\hat{P}}

% Domains
\newcommand{\domp}{\Omega_{p}}
\newcommand{\domu}{\Omega_{\ub}}
\newcommand{\domv}{\Omega_{\vb}}
\newcommand{\domui}{\Omega_{\bar{\mathbf{u}}_i}}
\newcommand{\domuj}{\Omega_{\bar{\mathbf{u}}_j}}
\newcommand{\gamja}{\Gamma_{j0}}
\newcommand{\gamjb}{\Gamma_{j1}}
\newcommand{\gamia}{\Gamma_{i0}}
\newcommand{\gamib}{\Gamma_{i1}}

% Comments
\newcommand{\red}{\color{red} AP:}
\newcommand{\usecomments}{true}
\newcommand{\ocomment}[1] {
\ifthenelse{ \equal{\usecomments}{true} }{
 \textbf{Ossian: }{\color{blue} #1}
}{
}
}



\begin{document}

%\title{The Lindblad master equation}

%% \author{N. Anders Petersson\thanks{Center for Applied
%%     Scientific Computing, Lawrence Livermore National Laboratory, L-561, PO Box 808, Livermore CA
%%     94551. }}

%% \date{\today}

%% \maketitle
\section{Ensambles of quantum states}

This note closely follows the presentation in Section 2.4.2 of Nielsen and
Chuang~\cite{Nielsen-Chuang}.  The probablity density operator (or matrix) is used to characterize
quantum systems whose state is not completely known. Suppose a quantum system is in one of a number
of states $\psi_i\in{\mathbb C}^N$ with respective probablility $p_i$, for $i\in[1,N]$. We call
$\{p_i, \psi_i\}$ an ensamble of pure states. The density matrix for the quantum system is defined
by the $N\times N$ complex matrix
\[
\rho = \sum_{i=1}^N p_i \psi_i \psi_i^\dag,\quad p_i \geq 0.
\]
From the definition follows immediately that the density matrix is Hermitian (self-adjoint),
\[
\rho^\dag = \sum_{i=1}^N p_i \psi_i \psi_i^\dag = \rho,
\]
because the probabilities $p_i$ are real non-negative numbers.

Consider the evolution of a closed quantum system that is described by the unitary operator $U$. Let
the system initially be in the state $\psi_i$, with probablitity $p_i$, corresponding to the density
matrix $\rho(0)$. After the evolution (at time $T$) the system will then be in the state $U\psi_i$
with probability $p_i$. Thus, the evolution of the density matrix is described by
\[
\rho(T) = \sum_{i=1}^N p_i U \psi_i \psi_i^\dag U^\dag =  U \rho(0) U^\dag.
\]

A quantum system whose state is known exactly is said to be in a pure state. Let the state be
represented by the unit vector $\psi\in C^N$. Because the system is in this state with certainty
(probability 1), the density matrix is represented by the rank-1 matrix
\[
\rho = \psi \psi^\dag \in {\mathbb C}^{N\times N}.
\]
In this case, the density matrix is a projection,
\[
\rho^2 = \psi \psi^\dag \psi\psi^\dag = \psi\psi^\dag = \rho.
\]

If the state of the quantum system is {\em not} known exactly, the density operator is in a mixed state, meaning that it
is in an ensamble of pure states that result in a density matrix that is not a projection.
\begin{definition}
  Consider a quantum system that is in a pure or a mixed state. Let $\psi_i\in{\mathbb C}^N$ be the
  state vector for the $i^{th}$ state and let the quantum system be in this state with probability
  $p_i\geq 0$, for $i\in[1,N]$. Also assume $\sum_{i=1}^N p_i = 1$. The Hermitian $N\times N$ probability density matrix
  for the quantum system is defined by
  \[
  \rho = \sum_{i=1}^N p_i \psi_i \psi_i^\dag,\quad p_i \geq 0,\quad \rho^\dag = \rho,\quad \mbox{tr$(\rho)$} = 1.
  \]
  Conversely, $\rho\in{\mathbb C}^{N\times N}$ is the probability density matrix associated with an ensamble of
  quantum states, if and only if:
  \[
  \rho^\dag = \rho,\quad \mbox{tr$(\rho)$} = 1,\quad
  \phi^\dag \rho \phi \geq 0,\quad \mbox{for all $\phi\in {\mathbb C}^N$}.
  \]
\end{definition}
%
A simple criteria for determining whether a state is pure or mixed is:
\[
\begin{cases} \mbox{tr$(\rho^2)= 1$},&\mbox{pure state}\\  \mbox{tr$(\rho^2)< 1$},&\mbox{mixed state} \end{cases}
\]
It is easy to see that the statement is true for a pure state, because then the quantum system is
described by the state vector $\psi$ with certainty. The corresponding density matrix becomes
\[
\rho = \psi \psi^\dag,\quad \Rightarrow \quad \rho^2 = \psi \psi^\dag \psi \psi^\dag = \psi \psi^\dag = \rho,
\]
because $\psi^\dag \psi = |\psi|^2 = 1$. Thus, $\mbox{tr$(\rho^2)$} = \mbox{tr$(\rho)= 1$}$.

\paragraph{Freedom of ensambles.} In this example, it is shown that two different ensambles of
quantum states can result in the same density matrix. Let $\eb_k$ be the $k$th unit vector, i.e.,
$\eb_0=[1,0,0,\ldots,0]^\dag$, $\eb_1=[0,1,0,\ldots,0]^\dag$, etc., and consider a quantum system
with the density matrix
\[
\rho = \frac{3}{4}\eb_0 \eb_0^\dag + \frac{1}{4}\eb_1 \eb_1^\dag.
\]
Clearly,  this density matrix arises if the quantum system is in the state $\eb_0$ with
probability $3/4$ and in state $\eb_1$ with probability $1/4$. Now define the states
\begin{align}
  \ab &= \sqrt{\frac{3}{4}} \eb_0 +  \sqrt{\frac{1}{4}} \eb_1, \\
  \bb &= \sqrt{\frac{3}{4}} \eb_0 -  \sqrt{\frac{1}{4}} \eb_1,
\end{align}
and let the quantum system be in the state $\ab$ with probability $1/2$ and in the state
$\bb$ with probability $1/2$. Then the corresponding density matrix becomes
\[
\rho = \frac{1}{2}\ab\ab^\dag + \frac{1}{2}\bb\bb^\dag =
%
\frac{1}{2}
\begin{bmatrix}
  \frac{3}{4} & \frac{\sqrt{3}}{4} \\
  \frac{\sqrt{3}}{4} & \frac{1}{4}
  \end{bmatrix} +
%
\frac{1}{2}\begin{bmatrix}
  \frac{3}{4} & -\frac{\sqrt{3}}{4} \\
  -\frac{\sqrt{3}}{4} & \frac{1}{4}
  \end{bmatrix} =
%
\begin{bmatrix}
  \frac{3}{4} & 0 \\
  0 & \frac{1}{4}
  \end{bmatrix}.
\]
Thus, two different ensambles of quantum states can result in the same density matrix! In general,
the eigenvectors and eigenvalues of a density matrix only indicate one of many possible ensambles
that result in the same density matrix. It is possible to further characterize the freedom of
ensambles that result in the same density operator, see~\cite{Nielsen-Chuang}.


\section{The Lindblad master equation}

Interactions with the environment in a quantum system can be modeled
by considering the evolution of the Hermitian density matrix,
\[
\rho =
\begin{bmatrix}
\rho_{11} & \rho_{12} & \ldots & \rho_{1N} \\
\rho_{21} & \rho_{22} & \ldots & \rho_{2N} \\
\vdots    & \vdots    &           & \vdots \\
\rho_{N1} & \rho_{N2} & \ldots & \rho_{NN}
\end{bmatrix}
%
= \rho^\dag.
\]
For a closed quantum system, the density matrix evolves according to
\begin{equation}\label{eq_LvN}
\dot{\rho} = -i\left(H\rho - \rho H\right),\quad t\geq 0, \quad
\rho(0) = \rho_0,\quad \mbox{tr$(\rho_0)$} = 1,\quad \rho_0^\dag = \rho_0.
\end{equation}
Here, $H = H(t)$ is the Hamiltonian matrix, $H\in{\mathbb C}^{N\times N}$ and $H^\dag = H$. Thus,
the expression on the right hand side satisfies
\[
\left( -i\left(H\rho - \rho H\right) \right)^\dag = +i \left( \rho^\dag H^\dag - H^\dag \rho^\dag
\right)
%
= +i \left( \rho H - H \rho \right),
\]
because both $H$ and $\rho$ are Hermitian. We conclude that the right hand side of \eqref{eq_LvN} is
anti-Hermitian (skew-symmetric). The evolution of the density matrix is therefore oscillatory in
time.

The above equation is called the Liouville-von Neumann (LvN) equation and is a special case of the
Lindblad master equation,
\begin{equation}\label{eq_Lindblad}
\dot{\rho} = -i\left(H\rho - \rho H\right) + \sum_j \gamma_j\left( {\cal L}_j \rho {\cal L}_j^\dag -
\frac{1}{2}\left( {\cal L}_j^\dag{\cal L}_j\rho + \rho{\cal L}_j^\dag{\cal L}_j \right) \right),
\end{equation}
where $\gamma_j$ are constants and ${\cal L}_j$ are collapse operators. One example of a collapse
operator is ${\cal L}_j = a$, which lowers the energy level. It is called a jump operator. Another
example is ${\cal L}_j = a^\dag a$. It is called a de-phasing operator. From the grouping of the
collapse operators in \eqref{eq_Lindblad}, we immediately see that each term in the sum on the right
hand side is Hermitian,
\begin{multline*}
\left( {\cal L}_j \rho {\cal L}_j^\dag -
\frac{1}{2}\left( {\cal L}_j^\dag{\cal L}_j\rho + \rho{\cal L}_j^\dag{\cal L}_j \right) \right)^\dag
=
%
 {\cal L}_j \rho^\dag {\cal L}_j^\dag -
 \frac{1}{2}\left( \rho^\dag{\cal L}_j^\dag{\cal L}_j + {\cal L}_j^\dag{\cal L}_j\rho^\dag \right) \\
 %
= {\cal L}_j \rho {\cal L}_j^\dag - \frac{1}{2}\left( {\cal L}_j^\dag{\cal L}_j\rho + \rho{\cal L}_j^\dag{\cal L}_j \right),
\end{multline*}
because $\rho^\dag = \rho$. Thus, the collapse operators act as damping or source terms in the
evolution of the density matrix.

\subsection{Vectorizing the density matrix}

The right hand sides of the LvN \eqref{eq_LvN} and Lindblad equations \eqref{eq_Lindblad} involve
the dependent variable $\rho$ in matrix form. To utilize the sparse structure of the
Hamiltonian matrix and use standard solvers for systems of ordinary differential equations, we
reorder $\rho$ by stacking its columns into one column vector with $N^2$ elements and use Kronecker
products to reformulate the equations. For the LvN equation, we get
\begin{equation}\label{eq_LvN-vec}
%
\frac{d}{dt}\mbox{vec$(\rho)$} = -i\left( I_N\otimes H - H^T \otimes I_N \right)\mbox{vec$(\rho)$},\quad
%
\mbox{vec$(\rho)$} =
\begin{bmatrix}
  \rho_{11}\\
  \vdots \\
  \rho_{N1} \\
%  \rho_{12} \\
%  \vdots \\
%  \rho_{N2} \\
  \vdots\\
  \vdots\\
  \rho_{1N} \\
  \vdots \\
  \rho_{NN}
\end{bmatrix}.
\end{equation}
Here, $I_N$ denotes the $N\times N$ identity matrix. Note that $H$ is transposed in the second term
on the right hand side (without complex conjugation). Because $H$ is Hermitian, the terms in the
matrix on the right hand side satisfy
\[
( I_N\otimes H )^\dag = I_N \otimes H^\dag = I_N \otimes H,\quad
%
(H^T \otimes I_N)^\dag = \bar{H} \otimes I_N = \bar{H}^\dag \otimes I_N = H^T \otimes I_N.
\]
Because of the factor $i$ in front of the matrix in \eqref{eq_LvN-vec}, it satisfies
\[
\left( -i\left( I_N\otimes H - H^T \otimes I_N \right) \right)^\dag = +i\left( I_N\otimes H - H^T \otimes I_N \right).
\]
This verifies that the skew-symmetric property of the LvN equation was retained after $\rho$ was
reordered into a column vector.

\subsection{Test problem 1}

Consider the LvN equation \eqref{eq_LvN} when the  Hamiltonian matrix and the initial data for the density matrix are
\[
H(t) = f(t)\begin{bmatrix} 0 & 1 \\ 1 & 0 \end{bmatrix},\quad
%
f(t) = \frac{1}{4}\left( 1 - \cos(\omega t) \right),\quad
%
\rho_0 = \begin{bmatrix} 1 & 0 \\ 0 & 0 \end{bmatrix}
\]
After vectorizing the problem, we get the system
\begin{equation}\label{eq_LvN-test1}
\frac{d}{dt}\mbox{vec$(\rho)$} = -i f(t)\begin{bmatrix}
  0 & 1 & -1 & 0 \\
  1 & 0 & 0 & -1 \\
 -1 & 0 & 0 &  1 \\
  0 & -1 & 1 & 0 
  \end{bmatrix}\mbox{vec$(\rho)$}
\end{equation}
Note that the matrix on the right hand side is skew-symmetric. Thus all eigenvalues are purely
imaginary and the solution is oscillatory in time. An exact solution can be constructed from the
Schrodinger equation with the same Hamiltonian matrix. It is solved by the state vector
\[
\Psi(t) = \begin{bmatrix}
\cos(\phi(t)) \\ -i \sin(\phi(t))
  \end{bmatrix},\quad \phi(t) = \frac{1}{4}\left( t - \frac{1}{\omega}\sin(\omega t) \right)
\]
The solution of \eqref{eq_LvN-test1} follows as the corresponding density matrix,
\[
\rho(t) = \Psi(t) \Psi^\dag(t) = \begin{bmatrix}
  \cos^2(\phi(t)) & i \cos(\phi(t)) \sin(\phi(t)) \\
  -i \cos(\phi(t)) \sin(\phi(t)) & \sin^2(\phi(t))
  \end{bmatrix}
\]
To further utilize standard routines for solving ordinary differential equations, we further re-formulate
the problem in terms of real-valued vectors and matrices,
\[
\mbox{vec$(\rho)$} = \ub + i \vb, \quad  -i\left( I_N\otimes H(t) - H^T(t) \otimes I_N \right) =:
A(t) + i B(t).
\]
This leads to the equivalent system,
\[
\frac{d}{dt}\begin{bmatrix} \ub \\ \vb \end{bmatrix} =
%
\begin{bmatrix}
  A(t) & -B(t) \\ B(t) & A(t)
\end{bmatrix}
\begin{bmatrix} \ub \\ \vb \end{bmatrix}
\]
Because the system \eqref{eq_LvN-vec} is skew-symmetric, $(A + i B)^\dag = - (A + i B)$. Thus, $A^T=-A$ and $B^T =
B$. As expected, the real-valued formulation is also skew-symmetric. Since $\rho$ is
Hermitian, not all off-diagonal elements are unique,
\[
\rho_{jk} = \bar{\rho}_{kj},\quad \Rightarrow\quad u_{jk} = u_{kj},\quad v_{jk} = - v_{kj}.
\]
Hence, only the lower triangular part of the density matrix needs to be stored for the real part of
$\rho$. Further savings can be realized for the imaginary part of $\rho$, because it is zero
along the diagonal, $v_{jj}=0$.
%\begin{itemize}
%  \item Show that the matrix is always skew-symmetric and that the solution is oscillatory in nature.
%  \item Present the density matrix formulation for the previous $2\time 2$ test problems for
%    Schrodinger's equation.
%  \item Note that the two test problems have identical density matrices at certain times.
%\end{itemize}

\bibliographystyle{plain}
\bibliography{quantum}


\end{document}
