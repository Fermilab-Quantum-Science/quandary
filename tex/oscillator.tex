\documentclass{beamer}
\usepackage{amsmath, graphicx, listings, color}

%% \setlength{\topmargin}{0 mm}
%% \setlength{\oddsidemargin}{5 mm}
%% \setlength{\evensidemargin}{5 mm}
%% \setlength{\textwidth}{150 mm}
%% \setlength{\textheight}{210 mm}

%
\newcommand{\calo}{{\cal O}}
%
\newcommand{\ab}{{\bf a}}
\newcommand{\bb}{{\bf b}}
\newcommand{\db}{{\bf d}}
\newcommand{\eb}{{\bf e}}
\newcommand{\fb}{{\bf f}}
\newcommand{\gb}{{\bf g}}
\newcommand{\hb}{{\bf h}}
\newcommand{\ib}{{\bf i}}
\newcommand{\jb}{{\bf j}}
\newcommand{\nb}{{\bf n}}
\newcommand{\pb}{{\bf p}}
\newcommand{\tb}{{\bf t}}
\newcommand{\rb}{{\bf r}}
\newcommand{\xb}{{\bf x}}
\newcommand{\yb}{{\bf y}}
\newcommand{\zb}{{\bf z}}
\newcommand{\qb}{{\bf q}}
\newcommand{\ub}{{\bf u}}
\newcommand{\vb}{{\bf v}}
\newcommand{\wb}{{\bf w}}

\newcommand{\Ab}{{\bf A}}
\newcommand{\Bb}{{\bf B}}
\newcommand{\Eb}{{\bf E}}
\newcommand{\Fb}{{\bf F}}
\newcommand{\Ib}{{\bf I}}
\newcommand{\Hb}{{\bf H}}
\newcommand{\Kb}{{\bf K}}
\newcommand{\Lb}{{\bf L}}
\newcommand{\Pb}{{\bf P}}
\newcommand{\Qb}{{\bf Q}}
\newcommand{\Rb}{{\bf R}}
\newcommand{\Ub}{{\bf U}}
\newcommand{\Tb}{{\bf T}}
\newcommand{\Xb}{{\bf X}}

\newcommand{\uh}{\hat{u}}
\newcommand{\vh}{\hat{v}}
\newcommand{\ph}{\hat{p}}
\newcommand{\qh}{\hat{q}}

\newcommand{\re}{{\rm Re}\,}
\newcommand{\im}{{\rm Im}\,}

\renewcommand{\arraystretch}{1.3}
%
\newcommand{\p}{\partial}
%
\newcommand{\eq}{\!\!\! = \!\!\!}
\newcommand{\om}{\omega}
\newcommand{\divergence}{\nabla\cdot}
\newcommand{\curl}{\nabla\times}

\newcommand{\alphab}{\boldsymbol{\alpha}}
\newcommand{\rhob}{\boldsymbol{\rho}}
\newcommand{\kab}{\boldsymbol{\kappa}}
\newcommand{\lambdab}{\boldsymbol{\lambda}}
\newcommand{\etab}{\boldsymbol{\eta}}
\newcommand{\psib}{\boldsymbol{\psi}}
\newcommand{\Psib}{\boldsymbol{\Psi}}
\newcommand{\zetab}{\boldsymbol{\zeta}}
\newcommand{\sigmab}{\boldsymbol{\sigma}}
\newcommand{\omegab}{\boldsymbol{\omega}}
\newcommand{\Gb}{{\bf G}}
\newcommand{\kb}{{\bf k}}
\newcommand{\sbold}{{\bf s}}

\title{Quantum harminic oscillator}
\author{N. Anders Petersson}
\institute{Lawrence Livermore National Laboratory\footnote{LLNL-PRES-abcdef;
This work was performed under the auspices of the U.S. Department of
Energy by Lawrence Livermore National Laboratory under Contract DE-AC52-07NA27344. Lawrence Livermore National Security, LLC.}}
\date{\today}

\begin{document}
%\lstset{language=[03]Fortran}
% to show file name, caption on bottom, framed.
% How to change the caption label further?
\lstset{
  caption=\lstname, captionpos=t, frame=single, basicstyle=\ttfamily\scriptsize,
  columns=fullflexible, keywordstyle=\color{blue},
  commentstyle=\color{red}, stringstyle=\color{gray},
  showspaces=false,
  showstringspaces=false, breaklines=true
}
\renewcommand\lstlistingname{File}
\renewcommand{\thelstlisting}{}
%%%%%%%%%%%%%%%
\frame{\titlepage}

%%%%%%%%%%%%%%%
\begin{frame}{Notation and lowering operator}
  Ground state eigenfunction $\psi_0(x)$ and first excited state $\psi_1(x)$ using vector and
  bra-ket notation
  \[
  \psi_0(x) = |0\rangle = \begin{bmatrix} 1\\ 0\\ 0
    \\ \vdots \end{bmatrix},
  \quad 
  \psi_1(x) = |1\rangle = \begin{bmatrix} 0\\ 1\\ 0
    \\ \vdots \end{bmatrix},
  \]
  The $n^{th}$ excited state eigenfunction: $\psi_n(x) = | n\rangle$.

  The lowering operator satisfies $a\psi_n(x) = \sqrt{n} \psi_{n-1}(x)$. In matrix form,
  \[
  a =  \begin{bmatrix}
    0 & 1 &  &  & \\
       & 0 & \sqrt{2} &  & \\
       &    & 0 &\sqrt{3} & \\
       &    &    & \ddots & \ddots
    \end{bmatrix}, \quad   a | n\rangle = \sqrt{n} | n-1\rangle.
  \]
\end{frame}


%%%%%%%%%%%%%%%%%%%%%%
\begin{frame}{Raising and number operators}

  The raising operator satisfies $a^\dag \psi_n(x) = \sqrt{n+1} \psi_{n+1}(x)$,
  \[
  a^\dag =  \begin{bmatrix}
    0 &  &  &  & \\
   1  & 0 &  &  & \\
      &  \sqrt{2}  & 0 & & \\
      &   &   \sqrt{3} & 0 & \\
      &   &  & \ddots & \ddots
    \end{bmatrix},\quad a^\dag | n\rangle  = \sqrt{n+1} | n+1\rangle.
  \]

The number operator, $N$,
\[
N = a^\dag a =
\begin{bmatrix}
   0 &    &  &  & \\
      & 1 &     &  & \\
      &   &  2 &    & \\
      &   &     & 3 & \\
      &   &     &  & \ddots
\end{bmatrix},\quad N | n\rangle = n | n \rangle.
\]
also, $a a^\dag = N + I$.
\end{frame}

%%%%%%%%%%%%%%%%%%%%%%
\begin{frame}{Hamiltonian and energy}
  Hamiltonian of a quantum harmonic oscillator in operator form,
  \[
  H = \hbar\omega\left( a^\dag a + \frac{1}{2} \right) = \hbar\omega\left( N + \frac{1}{2} I \right)
  \]
  and in matrix form
  \[
H = \frac{\hbar\omega}{2}
\begin{bmatrix}
   1 &    &  &  & \\
      & 3 &     &  & \\
      &   &  5 &    & \\
      &   &     & 7 & \\
      &   &     &  & \ddots
\end{bmatrix}
\]
The Hamiltonian of and eigenfunction $\psi_n(x)$,
\[
H \psi_n = \hbar\omega\left( N + \frac{1}{2} I \right) \psi_n = \hbar\omega\left( n + \frac{1}{2} \right) \psi_n,
\]
gives the energy level, $E_n = \hbar\omega\left( n + \frac{1}{2} \right)$.
\end{frame}

%%%%%%%%%%%%%%%%%%%%%%
\begin{frame}{Coupled oscillators}
  The system Hamiltonian for $Q$ qudits is
  \[
  H_0 = \sum_{j=0}^{Q-1} \left( \omega_j a^\dag_j a_j - \chi_{jj} a^\dag_ja^\dag_j a_j a_j -
  \sum_{k\ne j} \chi_{jk}  a^\dag_j  a_j  a^\dag_k  a_k \right)
  \]
  If we retain $L$ levels in each qudit, the lowering operators are defined in terms of the $L$ by
  $L$ identity matrix $I_L$. If $Q=2$,
  \[
  a_0 = a \otimes I_L, \quad  a_1 = I_L \otimes a.
  \]
  Thus,
  \begin{align*}
  a^\dag_0 a_0 &= (a^\dag \otimes I_L) (a \otimes I_L) = (a^\dag a)\otimes (I_L I_L) = N\otimes I_L,\\
  a^\dag_1 a_1 &= (I_L \otimes a^\dag ) (I_L \otimes a ) = (I_L I_L)  \otimes (a^\dag a) = I_L
  \otimes N.
  \end{align*}
  The matrices $a^\dag_0$ and $a^\dag_1$ have size $L^2$ by $L^2$.
\end{frame}

%%%%%%%%%%%%%
\begin{frame}{The system Hamiltonian}
  The matrices
  \begin{align*}
  a^\dag_0 a_0 & = N\otimes I_L =: N_0,\\
  a^\dag_1 a_1 & = I_L \otimes N =: N_1,
  \end{align*}
  are both diagonal. Furthermore,
  \begin{align*}
  a^\dag_0 a^\dag_0 a_0 a_0 &= N_0 N_0 - N_0,\\
  a^\dag_1 a^\dag_1 a_1 a_1 &= N_1 N_1 - N_1,
  \end{align*}
  are also diagonal. Thus all terms in the system Hamiltonian
  \[
  H_0 = \sum_{j=0}^{Q-1} \left( \omega_j N_j - \chi_{jj} (N_j^2 - N_j) -
  \sum_{k\ne j} \chi_{jk}  N_j  N_k \right)
  \]
  are diagonal.
  
\end{frame}

%%%%%%%%%%%%%
\begin{frame}{The control Hamiltonian}
  The control Hamiltonian satisfies
  \[
  H_c(t) = \sum_{k=0}^{Q-1} F(t)\left( a^\dag_k + a_k \right) +  i \sum_{k=0}^{Q-1} G(t)\left( a^\dag_k - a_k \right),
  \]
  where $F$ and $G$ are real-valued functions. These matrices are not diagonal, but block-diagonal. For example,
  \begin{multline*}
    a^\dag_0 + a_0 = (a\otimes I_L) + (a^\dag\otimes I_L) =\\
    %
    \begin{bmatrix}
    0 &  I_L &  &  & \\
    I_L  & 0 &  \sqrt{2}I_L &  & \\
      &  \sqrt{2}I_L  & 0 & \sqrt{3}I_L & \\
      &   &   \sqrt{3}I_L & 0 & \ddots \\
      &   &  & \ddots & \ddots
    \end{bmatrix}
  \end{multline*}
\end{frame}

\begin{frame}{The control Hamiltonian 2}
  Similarly,
  \begin{multline*}
    a^\dag_1 + a_1 = (I_L\otimes a) + (I_L \otimes a^\dag) =\\
    %
    \begin{bmatrix}
    a^\dag + a & 0 &  &  & \\
    0  & a^\dag + a &  0 &  & \\
      &  0 & a^\dag + a & 0 & \\
      &   &   0 & a^\dag + a & \ddots \\
      &   &  & \ddots & \ddots
    \end{bmatrix}
  \end{multline*}
  is a block-diagonal matrix.
\end{frame}

\end{document}

