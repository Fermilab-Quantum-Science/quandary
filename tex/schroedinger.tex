\documentclass[11pt]{article}
\usepackage{amsfonts,amsmath,amssymb,amsthm,graphicx}
\usepackage{multirow}
\usepackage{booktabs}
\usepackage[caption=false]{subfig} 
\usepackage{color}
\usepackage{ifthen}

\setlength{\topmargin}{0 mm}
\setlength{\oddsidemargin}{5 mm}
\setlength{\evensidemargin}{5 mm}
\setlength{\textwidth}{150 mm}
\setlength{\textheight}{210 mm}

\newtheorem{lemma}{Lemma}
\newtheorem{theorem}{Theorem}
\newtheorem{remark}{Remark}
\newcommand{\dzx}{D_0^x}
\newcommand{\dzxp}{D_{0p}^x}
\newcommand{\wdzx}{\widetilde{D_0^x}}
\newcommand{\wdzxp}{\widetilde{D_{0p}^x}}
\newcommand{\dpx}{D_+^x}
\newcommand{\dmx}{D_-^x}
\newcommand{\dzy}{D_0^y}
\newcommand{\dzyp}{D_{0p}^y}
\newcommand{\wdzy}{\widetilde{D_0^y}}
\newcommand{\dpy}{D_+^y}
\newcommand{\dmy}{D_-^y}
\newcommand{\dzz}{D_0^z}
\newcommand{\wdzz}{\widetilde{D_0^z}}
\newcommand{\dpz}{D_+^z}
\newcommand{\dmz}{D_-^z}
\newcommand{\dzt}{D_0^t}
\newcommand{\dpt}{D_+^t}
\newcommand{\dmt}{D_-^t}
\newcommand{\ehx}{E_{1/2}^x}
\newcommand{\ehy}{E_{1/2}^y}
\newcommand{\ehz}{E_{1/2}^z}
%
\newcommand{\calo}{{\cal O}}
%
\newcommand{\ab}{{\mathbf a}}
\newcommand{\bb}{{\mathbf b}}
\newcommand{\db}{{\mathbf d}}
\newcommand{\eb}{{\mathbf e}}
\newcommand{\fb}{{\mathbf f}}
\newcommand{\gb}{{\mathbf g}}
\newcommand{\ib}{{\mathbf i}}
\newcommand{\jb}{{\mathbf j}}
\newcommand{\nb}{{\mathbf n}}
\newcommand{\pb}{{\mathbf p}}
\newcommand{\tb}{{\mathbf t}}
\newcommand{\rb}{{\mathbf r}}
\newcommand{\yb}{{\mathbf y}}
\newcommand{\zb}{{\mathbf z}}
\newcommand{\qb}{{\mathbf q}}
\newcommand{\ub}{{\mathbf u}}
\newcommand{\vb}{{\mathbf v}}
\newcommand{\wb}{{\mathbf w}}

\newcommand{\Ab}{{\mathbf A}}
\newcommand{\Bb}{{\mathbf B}}
\newcommand{\Eb}{{\mathbf E}}
\newcommand{\Fb}{{\mathbf F}}
\newcommand{\Ib}{{\mathbf I}}
\newcommand{\Hb}{{\mathbf H}}
\newcommand{\Kb}{{\mathbf K}}
\newcommand{\Lb}{{\mathbf L}}
\newcommand{\Pb}{{\mathbf P}}
\newcommand{\Qb}{{\mathbf Q}}
\newcommand{\Rb}{{\mathbf R}}
\newcommand{\Ub}{{\mathbf U}}
\newcommand{\Tb}{{\mathbf T}}
\newcommand{\Xb}{{\mathbf X}}

\newcommand{\Abb}{\mathbb{A}}
\newcommand{\Bbbb}{\mathbb{B}}
\newcommand{\Ebb}{\mathbb{E}}
\newcommand{\Fbb}{\mathbb{F}}
\newcommand{\Ibb}{\mathbb{I}}
\newcommand{\Hbb}{\mathbb{H}}
\newcommand{\Kbb}{\mathbb{K}}
\newcommand{\Lbb}{\mathbb{L}}
\newcommand{\Pbb}{\mathbb{P}}
\newcommand{\Qbb}{\mathbb{Q}}
\newcommand{\Rbb}{\mathbb{R}}
\newcommand{\Ubb}{\mathbb{U}}
\newcommand{\Tbb}{\mathbb{T}}
\newcommand{\Xbb}{\mathbb{X}}
\newcommand{\Ybb}{\mathbb{Y}}
\newcommand{\Zbb}{\mathbb{Z}}

\newcommand{\uh}{\hat{u}}
\newcommand{\vh}{\hat{v}}
\newcommand{\ph}{\hat{p}}
\newcommand{\qh}{\hat{q}}

\newcommand{\re}{{\rm Re}\,}
\newcommand{\im}{{\rm Im}\,}

\renewcommand{\arraystretch}{1.3}
%
\newcommand{\p}{\partial}
%
\newcommand{\eq}{\!\!\! = \!\!\!}
\newcommand{\om}{\omega}
%\newcommand{\divergence}{\nabla\cdot}
%\newcommand{\curl}{\nabla\times}

\newcommand{\alphab}{\boldsymbol{\alpha}}
\newcommand{\lambdab}{\boldsymbol{\lambda}}
\newcommand{\psib}{\boldsymbol{\psi}}
\newcommand{\phib}{\boldsymbol{\phi}}
\newcommand{\Psib}{\boldsymbol{\Psi}}

\newcommand{\rhob}{\boldsymbol{\rho}}
\newcommand{\kab}{\boldsymbol{\kappa}}
\newcommand{\etab}{\boldsymbol{\eta}}
\newcommand{\zetab}{\boldsymbol{\zeta}}
\newcommand{\sigmab}{\boldsymbol{\sigma}}
\newcommand{\omegab}{\boldsymbol{\omega}}
\newcommand{\Gb}{{\mathbf G}}
\newcommand{\kb}{{\mathbf k}}
\newcommand{\sbold}{{\mathbf s}}
\newcommand{\ba}{\begin{array}}
\newcommand{\ea}{\end{array}}
\newcommand{\be}{\begin{equation}}
\newcommand{\ee}{\end{equation}}
\newcommand{\bd}{\begin{displaymath}}
\newcommand{\ed}{\end{displaymath}}
\newcommand{\pa}{\partial}
\newcommand{\f}{\frac}
\newcommand{\drp}{D^r_+}
\newcommand{\drm}{D^r_-}
\newcommand{\dqp}{D^q_+}
\newcommand{\dqm}{D^q_-}
\newcommand{\dtqn}{\widetilde{{D^q_0}} }
\newcommand{\dtrn}{\widetilde{{D^r_0}} }
\newcommand{\dqn}{D^q_0}
\newcommand{\drn}{D^r_0}
\newcommand{\erh}{E^r_{1/2}}
\newcommand{\eqh}{E^q_{1/2}}

\def\dpl{D_+}
\def\dmi{D_-}

\newcommand{\ubbar}{\bar{\mathbf{u}}}
\newcommand{\ubar}{\bar{u}}

% Numerical solutions
% \newcommand{\vb}{\mathbf{v}}


% Grids
\newcommand{\xb}{\mathbf{x}}
\newcommand{\ybh}{\hat{\mathbf{x}}}
\newcommand{\xbh}{\hat{\mathbf{x}}}
\newcommand{\Ja}{J_{\alpha}}
\newcommand{\ga}{g_{\alpha}}
\newcommand{\Ma}{M_{\alpha}}

% Interpolation
\newcommand{\Nxy}{N_{\mathbf{x}\rightarrow\hat{\mathbf{x}}}}
\newcommand{\Nyx}{N_{\hat{\mathbf{x}}\rightarrow\mathbf{x}}}
\newcommand{\Nuv}{N_{\bar{\ub}_1\rightarrow\bar{\ub}_2}}
\newcommand{\Nvu}{N_{\bar{\ub}_2\rightarrow\bar{\ub}_1}}
\newcommand{\Nij}{N_{\bar{\ub}_i\rightarrow\bar{\ub}_j}}
\newcommand{\Nji}{N_{\bar{\ub}_j\rightarrow\bar{\ub}_i}}
\newcommand{\Px}{P}
\newcommand{\Pxh}{\hat{P}}

% Domains
\newcommand{\domp}{\Omega_{p}}
\newcommand{\domu}{\Omega_{\ub}}
\newcommand{\domv}{\Omega_{\vb}}
\newcommand{\domui}{\Omega_{\bar{\mathbf{u}}_i}}
\newcommand{\domuj}{\Omega_{\bar{\mathbf{u}}_j}}
\newcommand{\gamja}{\Gamma_{j0}}
\newcommand{\gamjb}{\Gamma_{j1}}
\newcommand{\gamia}{\Gamma_{i0}}
\newcommand{\gamib}{\Gamma_{i1}}

% Comments
\newcommand{\red}{\color{red} AP:}
\newcommand{\usecomments}{true}
\newcommand{\ocomment}[1] {
\ifthenelse{ \equal{\usecomments}{true} }{
 \textbf{Ossian: }{\color{blue} #1}
}{
}
}



\begin{document}

\title{Notes on Schr\"odinger's equation}

\author{N. Anders Petersson\thanks{Center for Applied
    Scientific Computing, Lawrence Livermore National Laboratory, L-561, PO Box 808, Livermore CA
    94551. }}

\date{\today}

\maketitle

\section{Eigenfunctions of the continuous problem}
 Consider the 1-D Schr\"odinger equation governing the wave function $\Psi(x,t)$,
\begin{equation}\label{eq_shrodinger-time}
i\hbar\frac{\p \Psi}{\p t} = -\frac{\hbar^2}{2 m} \frac{\p^2}{\p x^2} \Psi + V(x) \Psi,\quad t\geq
0,\quad \| \Psi \| < \infty
\end{equation}
subject to appropriate initial conditions. Here, $V(x)$ is a potential function. For a harmonic
oscillator, the potential function satisfies
\[
V(x) = \frac{1}{2}m\omega^2 x^2.
\]
We can solve \eqref{eq_shrodinger-time} using separation of variables. We make the ansatz
\[
\Psi(x,t) = \phi(t) \psi(x),
\]
which we insert into \eqref{eq_shrodinger-time}. In the standard way, we obtain
\begin{equation} \label{eq_const-ratio}
E:= \frac{i\hbar \phi'}{\phi} = \frac{1}{\psi} \left(-\frac{\hbar^2}{2m} \frac{\p^2\psi}{\p x^2} +
\frac{1}{2} m \omega^2 x^2 \psi \right).
\end{equation}
Because $\phi=\phi(t)$ and $\psi=\psi(x)$, the coefficient $E$ must be constant for all $x$ and
$t$. This observation leads to the eigenvalue problem
\begin{equation}\label{eq_eigen-problem}
E\psi = -\frac{\hbar^2}{2m} \frac{\p^2\psi}{\p x^2} + \frac{1}{2} m \omega^2 x^2 \psi := \hat{H}\psi.
\end{equation}
Here, $E$ is the eigenvalue, $\psi$ is the eigenfunction, and $\hat{H}$ is called the Hamiltonian
operator. The right hand side can be factored according to
\[
\hat{H}\psi = \frac{\hbar\omega}{2}
%
\left(\sqrt{\frac{m\omega}{\hbar}} x - \frac{\hbar}{\sqrt{m\omega}} \frac{\p}{\p x} \right)
%
\left(\sqrt{\frac{m \omega}{\hbar}} x + \frac{\hbar}{\sqrt{m\omega}} \frac{\p}{\p x} \right) \psi +
%
\frac{\hbar \omega}{2} \psi.
\]
We can therefore factor the Hamiltonian operator as
\begin{equation}\label{eq_ladder}
\hat{H}\psi = \hbar\omega\left( a^\dag a + \frac{1}{2} I\right)\psi,
\end{equation}
where we have introduced the lowering operator $a$,
\begin{equation}
  a = \frac{1}{\sqrt{2}} \left(\frac{x}{x_0} + x_0 \frac{\p}{\p x} \right),\quad x_0 = \sqrt{\frac{\hbar}{m\omega}},
\end{equation}
and its adjoint, the raising operator, $a^\dag$,
\begin{equation}
  a^\dag= \frac{1}{\sqrt{2}} \left(\frac{x}{x_0} - x_0 \frac{\p}{\p x} \right).
\end{equation}

The factorization of the Hamiltonian operator allows the eigenvalues and eigenfunctions to be
computed in a very elegant way. The fundamental eigenfunction corresponds to the ground state and is
denoted $\psi_0(x)$. It satisfies
\[
a \psi_0 = 0.
\]
This equation is solved by the normalized Gaussian function
\[
\psi_0(x) = \frac{e^{-x^2/(2 x_0^2)}}{\pi^{1/4} x_0^{1/2}},\quad \int_{-\infty}^\infty |\psi_0|^2 \,
dx = 1.
\]
The corresponding eigenvalue follows from \eqref{eq_ladder},
\[
\hat{H}\psi_0 = \hbar\omega\left( a^\dag a + \frac{1}{2}I \right)\psi_0 = \frac{\hbar\omega}{2}
\psi_0,\quad E_0 = \frac{\hbar\omega}{2}.
\]

To derive the higher eigenfunctions, we start by studying the commutator
\begin{equation}\label{eq_a-commut}
[a,a^\dag] = a a^\dag - a^\dag a = \ldots = I.
\end{equation}
Therefore,
\[
  [\hat{H}a^\dag] = \hat{H}a^\dag - a^\dag \hat{H} = \hbar\omega(a^\dag a a^\dag - a^\dag a^\dag a) =
%
  \hbar\omega a^\dag[a,a^\dag] =   \hbar\omega a^\dag.
\]
We conclude that
\begin{equation}\label{eq_comm-rel}
\hat{H} a^\dag = a^\dag \hat{H} +  \hbar\omega a^\dag.
\end{equation}

Assume we know the normalized eigenfunction $\psi_n(x)$ with eigenvalue $E_n$. From
\eqref{eq_comm-rel},
\[
\hat{H} a^\dag\psi_n = a^\dag \hat{H} \psi_n + \hbar\omega a^\dag \psi_n = (E_n + \hbar\omega)
a^\dag \psi_n.
\]
We conclude that $a^\dag\psi_n$ is an (unnormalized) eigenfunction with
eigenvalue $E_{n+1} = E_n + \hbar\omega$. We write the normalized eigenfunction as
$\psi_{n+1} = \beta^{-1} a^\dag \psi_n$, where the normalization factor $\beta$ will be determined below.
Because $\psi_0$ has eigenvalue $E_0 = \hbar\omega/2$,
\begin{align*}
  E_1 &= E_0 + \hbar\omega = \hbar\omega\left( 1+\frac{1}{2} \right),\\
  E_2 &= E_1 + \hbar\omega = E_0 + 2\hbar\omega = \left( 2 + \frac{1}{2} \right),\\
  &\vdots\\
  E_n &= \hbar\omega\left( n + \frac{1}{2} \right),\quad n=0,1,2,\ldots.
\end{align*}
From the definition of the eigenvalue problem \eqref{eq_eigen-problem} and the factorization
\eqref{eq_ladder},
\[
\hbar\omega\left( a^\dag a + \frac{1}{2}I \right)\psi_n = E_n \psi_n = \hbar\omega\left( n +
\frac{1}{2} \right)\psi_n. 
\]
We define the number operator by
\begin{equation}\label{eq_numer-op}
\hat{N} = a^\dag a.
\end{equation}
It has the same eigenfunctions as $\hat{H}$ and non-negative integer eigenvalues,
\[
\hat{N}\psi_n = n \psi_n,\quad n=0,1,2,\ldots
\]

To normalize the eigenfunction $\psi_{n+1}(x) = \beta^{-1} a^\dag \psi_n$, we start by defining $u(x) =
a^\dag\psi_n(x)$. Its norm is defined by
\[
\| u \|^2 = \langle u, u \rangle = \int_{-\infty}^\infty \bar{u}(x) u(x)\, dx.
\]
From the definition of the adjoint of an operator,
\[
\langle a^\dag\psi_n, a^\dag\psi_n\rangle = \langle \psi_n, a a^\dag \psi_n \rangle
\]
From \eqref{eq_a-commut} and \eqref{eq_numer-op}, $a a^\dag = \hat{N} + I$. Therefore,
\[
\langle a^\dag\psi_n, a^\dag\psi_n\rangle = \langle \psi_n, (\hat{N}+I)\psi_n \rangle = (n+1) \| \psi_n \|^2.
\]
Because $\| \psi_n \|=1$, the normalized eigenfunction becomes
\begin{equation}\label{eq_eigenfunction-recursive}
\psi_{n+1}(x) = \frac{1}{\sqrt{n+1}} a^\dag\psi_n(x),\quad n=0,1,2,\ldots.
\end{equation}
This relation can also be written
\[
\sqrt{n+1}\, \psi_{n+1} = a^\dag \psi_n.
\]
By applying the lowering operator $a$ to the above equation,
\[
\sqrt{n+1} a\psi_{n+1} = a a^\dag \psi_n = (\hat{N}+I) \psi_n = (n+1) \psi_n.
\]
Thus, $a \psi_{n+1} = \sqrt{n+1} \psi_n$, i.e.,
\[
a \psi_n = \sqrt{n} \psi_{n-1}.
\]

By applying \eqref{eq_eigenfunction-recursive} recursively,
\[
\psi_{n}(x) = \frac{1}{\sqrt{n}}\,
(a^\dag)^n\psi_0(x),\quad n=1,2,3,\ldots. 
\]
The eigenfunctions can be expressed in terms of the $n^{th}$ order Hermite polynomials,
\[
\kappa_n(x) = (-1)^n e^{x^2} \frac{d^n}{dx^n} e^{-x^2}.
\]
By using Rodrigues formulae,
\begin{equation}\label{eq_eigenfunc-hermite}
\psi_n(x) = \frac{1}{\pi^{1/4} x_0^{n+1/2} \sqrt{2^n n!}} \kappa_n(x/x_0) e^{-x^2/(2 x_0^2)},\quad
x_0=\sqrt{\frac{\hbar}{m\omega}}.
\end{equation}

We summarize the most important relations in the following lemma.
\begin{lemma}
  The eigenvalue problem for the 1-D Schr\"odinger equation is
  \[
  E\psi = \frac{\hbar\omega}{2}\left( - x_0^2 \frac{\p^2\psi}{\p x^2} + \frac{x^2}{x_0^2}
  \psi\right) := \hat{H}\psi.
  \]
  Here, $x_0$ is a real constant, $E$ is the eigenvalue, $\psi$ is the eigenfunction, and $\hat{H}$
  is called the Hamiltonian operator. It can be factored into
  \[
  \hat{H} = \hbar\omega \left( a^\dag a + \frac{1}{2} \right),
  \]
  where $a$ and $a^\dag$ are called the lowering and raising operators, respectively,
  \[
  a = \frac{1}{\sqrt{2}}\left( \frac{x}{x_0} + x_0 \frac{\p}{\p x}\right),\quad
  %
  a^\dag = \frac{1}{\sqrt{2}}\left( \frac{x}{x_0} - x_0 \frac{\p}{\p x}\right),\quad
  x_0=\sqrt{\frac{\hbar}{m\omega}}.
  \]
  The smallest eigenvalue of $H$ and the corresponding normalized eigenfunction are
  \[
  E_0 = \frac{1}{2}\hbar\omega,\quad \psi_0(x) = \frac{e^{-x^2/(2 x_0^2)}}{\pi^{1/4} x_0^{1/2}}.
  \]
  The higher eigenvalues satisfy
  \[
  E_n = \hbar\omega\left(n + \frac{1}{2}\right),
  \]
  and the eigenfunctions are given by \eqref{eq_eigenfunc-hermite}. The normalized eigenfunctions
  satisfy the recursive relations
  \[
  a \psi_n = \sqrt{n} \psi_{n-1},\quad a^\dag \psi_n(x) = \sqrt{n+1}\, \psi_{n+1}.
  \]
\end{lemma}

From \eqref{eq_const-ratio}, the time dependence corresponding to the eigenfunction $\psi_n(x)$ with
eigenvalue $E_n$ satisfies 
\[
i\hbar \phi'_n = E_n \phi = \hbar\omega(n+1/2) \phi,
\]
which is solved by
\[
\phi_n(t) = c_n e^{-i(n+1/2)\omega t}.
\]
Thus, a general solution of the time-dependent Schr\"odinger equation can be written as an
eigenfunction expansion
\[
\Psi(x,t) = \sum_{n=0}^\infty c_n \psi_n(x) e^{-i(n+1/2)\omega t},
\]
where the coefficients $c_n$ are determined by the initial data.

\section{Matrix formulation}

\bibliographystyle{plain}
\bibliography{references}

\end{document}
