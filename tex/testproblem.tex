\documentclass[11pt]{article}
\usepackage{amsfonts,amsmath,amssymb,amsthm,graphicx}
\usepackage{multirow}
\usepackage{booktabs}
\usepackage[caption=false]{subfig} 
\usepackage{color}
\usepackage{ifthen}

\setlength{\topmargin}{0 mm}
\setlength{\oddsidemargin}{5 mm}
\setlength{\evensidemargin}{5 mm}
\setlength{\textwidth}{150 mm}
\setlength{\textheight}{210 mm}

\newtheorem{lemma}{Lemma}
\newtheorem{theorem}{Theorem}
\newtheorem{remark}{Remark}
\newcommand{\dzx}{D_0^x}
\newcommand{\dzxp}{D_{0p}^x}
\newcommand{\wdzx}{\widetilde{D_0^x}}
\newcommand{\wdzxp}{\widetilde{D_{0p}^x}}
\newcommand{\dpx}{D_+^x}
\newcommand{\dmx}{D_-^x}
\newcommand{\dzy}{D_0^y}
\newcommand{\dzyp}{D_{0p}^y}
\newcommand{\wdzy}{\widetilde{D_0^y}}
\newcommand{\dpy}{D_+^y}
\newcommand{\dmy}{D_-^y}
\newcommand{\dzz}{D_0^z}
\newcommand{\wdzz}{\widetilde{D_0^z}}
\newcommand{\dpz}{D_+^z}
\newcommand{\dmz}{D_-^z}
\newcommand{\dzt}{D_0^t}
\newcommand{\dpt}{D_+^t}
\newcommand{\dmt}{D_-^t}
\newcommand{\ehx}{E_{1/2}^x}
\newcommand{\ehy}{E_{1/2}^y}
\newcommand{\ehz}{E_{1/2}^z}
%
\newcommand{\calo}{{\cal O}}
%
\newcommand{\ab}{{\mathbf a}}
\newcommand{\bb}{{\mathbf b}}
\newcommand{\db}{{\mathbf d}}
\newcommand{\eb}{{\mathbf e}}
\newcommand{\fb}{{\mathbf f}}
\newcommand{\gb}{{\mathbf g}}
\newcommand{\ib}{{\mathbf i}}
\newcommand{\jb}{{\mathbf j}}
\newcommand{\nb}{{\mathbf n}}
\newcommand{\pb}{{\mathbf p}}
\newcommand{\tb}{{\mathbf t}}
\newcommand{\rb}{{\mathbf r}}
\newcommand{\yb}{{\mathbf y}}
\newcommand{\zb}{{\mathbf z}}
\newcommand{\qb}{{\mathbf q}}
\newcommand{\ub}{{\mathbf u}}
\newcommand{\vb}{{\mathbf v}}
\newcommand{\wb}{{\mathbf w}}

\newcommand{\Ab}{{\mathbf A}}
\newcommand{\Bb}{{\mathbf B}}
\newcommand{\Eb}{{\mathbf E}}
\newcommand{\Fb}{{\mathbf F}}
\newcommand{\Ib}{{\mathbf I}}
\newcommand{\Hb}{{\mathbf H}}
\newcommand{\Kb}{{\mathbf K}}
\newcommand{\Lb}{{\mathbf L}}
\newcommand{\Pb}{{\mathbf P}}
\newcommand{\Qb}{{\mathbf Q}}
\newcommand{\Rb}{{\mathbf R}}
\newcommand{\Ub}{{\mathbf U}}
\newcommand{\Tb}{{\mathbf T}}
\newcommand{\Xb}{{\mathbf X}}

\newcommand{\Abb}{\mathbb{A}}
\newcommand{\Bbbb}{\mathbb{B}}
\newcommand{\Ebb}{\mathbb{E}}
\newcommand{\Fbb}{\mathbb{F}}
\newcommand{\Ibb}{\mathbb{I}}
\newcommand{\Hbb}{\mathbb{H}}
\newcommand{\Kbb}{\mathbb{K}}
\newcommand{\Lbb}{\mathbb{L}}
\newcommand{\Pbb}{\mathbb{P}}
\newcommand{\Qbb}{\mathbb{Q}}
\newcommand{\Rbb}{\mathbb{R}}
\newcommand{\Ubb}{\mathbb{U}}
\newcommand{\Tbb}{\mathbb{T}}
\newcommand{\Xbb}{\mathbb{X}}
\newcommand{\Ybb}{\mathbb{Y}}
\newcommand{\Zbb}{\mathbb{Z}}

\newcommand{\uh}{\hat{u}}
\newcommand{\vh}{\hat{v}}
\newcommand{\ph}{\hat{p}}
\newcommand{\qh}{\hat{q}}

\newcommand{\re}{{\rm Re}\,}
\newcommand{\im}{{\rm Im}\,}

\renewcommand{\arraystretch}{1.3}
%
\newcommand{\p}{\partial}
%
\newcommand{\eq}{\!\!\! = \!\!\!}
\newcommand{\om}{\omega}
%\newcommand{\divergence}{\nabla\cdot}
%\newcommand{\curl}{\nabla\times}

\newcommand{\alphab}{\boldsymbol{\alpha}}
\newcommand{\lambdab}{\boldsymbol{\lambda}}
\newcommand{\psib}{\boldsymbol{\psi}}
\newcommand{\phib}{\boldsymbol{\phi}}
\newcommand{\Psib}{\boldsymbol{\Psi}}

\newcommand{\rhob}{\boldsymbol{\rho}}
\newcommand{\kab}{\boldsymbol{\kappa}}
\newcommand{\etab}{\boldsymbol{\eta}}
\newcommand{\zetab}{\boldsymbol{\zeta}}
\newcommand{\sigmab}{\boldsymbol{\sigma}}
\newcommand{\omegab}{\boldsymbol{\omega}}
\newcommand{\Gb}{{\mathbf G}}
\newcommand{\kb}{{\mathbf k}}
\newcommand{\sbold}{{\mathbf s}}
\newcommand{\ba}{\begin{array}}
\newcommand{\ea}{\end{array}}
\newcommand{\be}{\begin{equation}}
\newcommand{\ee}{\end{equation}}
\newcommand{\bd}{\begin{displaymath}}
\newcommand{\ed}{\end{displaymath}}
\newcommand{\pa}{\partial}
\newcommand{\f}{\frac}
\newcommand{\drp}{D^r_+}
\newcommand{\drm}{D^r_-}
\newcommand{\dqp}{D^q_+}
\newcommand{\dqm}{D^q_-}
\newcommand{\dtqn}{\widetilde{{D^q_0}} }
\newcommand{\dtrn}{\widetilde{{D^r_0}} }
\newcommand{\dqn}{D^q_0}
\newcommand{\drn}{D^r_0}
\newcommand{\erh}{E^r_{1/2}}
\newcommand{\eqh}{E^q_{1/2}}

\def\dpl{D_+}
\def\dmi{D_-}

\newcommand{\ubbar}{\bar{\mathbf{u}}}
\newcommand{\ubar}{\bar{u}}

% Numerical solutions
% \newcommand{\vb}{\mathbf{v}}


% Grids
\newcommand{\xb}{\mathbf{x}}
\newcommand{\ybh}{\hat{\mathbf{x}}}
\newcommand{\xbh}{\hat{\mathbf{x}}}
\newcommand{\Ja}{J_{\alpha}}
\newcommand{\ga}{g_{\alpha}}
\newcommand{\Ma}{M_{\alpha}}

% Interpolation
\newcommand{\Nxy}{N_{\mathbf{x}\rightarrow\hat{\mathbf{x}}}}
\newcommand{\Nyx}{N_{\hat{\mathbf{x}}\rightarrow\mathbf{x}}}
\newcommand{\Nuv}{N_{\bar{\ub}_1\rightarrow\bar{\ub}_2}}
\newcommand{\Nvu}{N_{\bar{\ub}_2\rightarrow\bar{\ub}_1}}
\newcommand{\Nij}{N_{\bar{\ub}_i\rightarrow\bar{\ub}_j}}
\newcommand{\Nji}{N_{\bar{\ub}_j\rightarrow\bar{\ub}_i}}
\newcommand{\Px}{P}
\newcommand{\Pxh}{\hat{P}}

% Domains
\newcommand{\domp}{\Omega_{p}}
\newcommand{\domu}{\Omega_{\ub}}
\newcommand{\domv}{\Omega_{\vb}}
\newcommand{\domui}{\Omega_{\bar{\mathbf{u}}_i}}
\newcommand{\domuj}{\Omega_{\bar{\mathbf{u}}_j}}
\newcommand{\gamja}{\Gamma_{j0}}
\newcommand{\gamjb}{\Gamma_{j1}}
\newcommand{\gamia}{\Gamma_{i0}}
\newcommand{\gamib}{\Gamma_{i1}}

% Comments
\newcommand{\red}{\color{red} AP:}
\newcommand{\usecomments}{true}
\newcommand{\ocomment}[1] {
\ifthenelse{ \equal{\usecomments}{true} }{
 \textbf{Ossian: }{\color{blue} #1}
}{
}
}



\begin{document}

%% \title{Notes on Schroedinger's equation}

%% \author{N. Anders Petersson\thanks{Center for Applied
%%     Scientific Computing, Lawrence Livermore National Laboratory, L-561, PO Box 808, Livermore CA
%%     94551. }}

%% \date{\today}

%% \maketitle
\section{Analytical solutions}
Consider the Schr\"odinger equations,
\begin{equation}\label{eq_shrodinger}
\dot{\psi} = -i H(t) \psi,\quad t\geq 0,\quad \psi(0) = \psi_0,
\end{equation}
where $\psi(t)\in {\mathbb C}^N$ and $H\in {\mathbb C}^{N\times N}$ is a Hermitian matrix.
\paragraph{Test problem 1.}
Consider the $2\times 2$ case with
\[
f(t) = \frac{1}{4}\left(1 - \cos(\omega t)\right),\quad H(t) = f(t) (a +a^\dag),\quad a+ a^\dag
=
\begin{bmatrix}
  0 & 1\\
  1 & 0
\end{bmatrix}.
\]
It has the analytical solution
\[
\psi(t) = \begin{bmatrix}
  \cos(\phi(t))\\
  -i\sin(\phi(t))
  \end{bmatrix},\quad \phi(t) = \frac{1}{4}\left( t - \frac{1}{\omega}\sin(\omega t)\right).
\]
% With the initial condition $\psi_0=[1, 0]^T$, and $\omega=2\pi$, the solution is given in Figure \ref{fig_2}.
  %% \begin{figure}
  %%   \begin{center}
  %%   \includegraphics[width=0.45\linewidth]{sep-test.png}
  %%   \includegraphics[width=0.45\linewidth]{nonsep-test.png}
  %%   \caption{Analytical solutions of test problem \#1 (left) and \#2 (right). Note that the imaginary part is zero for both
  %%     components of $\psi$ in the right figure.}\label{fig_2}
  %%   \end{center}
  %% \end{figure}

\paragraph{Test problem 2.}
By changing the Hamiltonian matrix to
\begin{equation}\
H(t) = i g(t) (a - a^\dag),\quad g(t) = \frac{1}{4}\left(1 - \sin(\omega t)\right),
\end{equation}
where
\[
a- a^\dag =
\begin{bmatrix}
  0  & 1\\
  -1 & 0
\end{bmatrix},
\]
the solution of the Schr\"odinger equation becomes
\[
\psi(t) = \begin{bmatrix}
  \cos(\theta(t))\\
  -\sin(\theta(t))
  \end{bmatrix},\quad \theta(t) = \frac{1}{4}\left( t + \frac{1}{\omega}(\cos(\omega t) - 1)\right).
\]
Note that both components of $\psi(t)$ are real in this case.
% With the initial condition $\psi_0=[1, 0]^T$, and $\omega=2\pi$, the solution is given in Figure~\ref{fig_2}.

\section{Real-valued formulation}

To utilize efficient numerical software it is desirable to derive a real-valued equivalent of
\eqref{eq_shrodinger}. A fairly general form of the Hamiltonian matrix is
\[
H(t) = H_d + f(t)(a+a^\dag) + i g(t)(a-a^\dag).
\]
Let the real-valued functions $u(t)$ and $v(t)$ hold the real and negative imaginary parts of $\psi(t)$,
\[
\psi(t) = u(t) - iv(t)
\]
and decompose the total Hamiltonian matrix into $H(t) = K(t) + i S(t)$, where the real-valued
matrices $K$ and $S$ hold the symmeric and skew-symmetric parts of $H$: $K^\dag = K$ and $S^\dag =
-S$. We have,
\begin{align*}
H\psi &= (K+iS)(u - iv) = (Ku + Sv) + i(Su - Kv),\\
-iH\psi &= -i(Ku + Sv) + (Su - Kv).
\end{align*}
Therefore, a real-valued equivalent of the system \eqref{eq_shrodinger} is
\begin{equation}\label{eq_real-shrodinger}
  \begin{bmatrix} \dot{u}\\ \dot{v} \end{bmatrix} =
%
  \begin{bmatrix}
    S(t) & -K(t) \\ K(t) & S(t)
  \end{bmatrix}     
  %
  \begin{bmatrix} u\\ v \end{bmatrix} .
\end{equation}
This is a Hamiltonian system where the time-dependent Hamiltonian function is
\begin{equation}\label{eq_hamiltonian}
\kappa(u,v,t) = u^T S(t) v + \frac{1}{2} u^T K(t) u + \frac{1}{2} v^T K(t) v.
\end{equation}
In general, the Hamiltonian function is non-separable due to the skew-symmetric term
$S(t) =g(t) ( a- a^\dag)$ in the Hamiltonian matrix.

\subsection{Time integration}
We consider integrating \eqref{eq_real-shrodinger} with a time-dependent forcing,
\begin{equation}\label{eq_s-v-forcing}
  \begin{bmatrix} \dot{u}\\ \dot{v} \end{bmatrix} =
%
  \begin{bmatrix}
    S(t) & -K(t) \\ K(t) & S(t)
  \end{bmatrix}     
  %
  \begin{bmatrix} u\\ v \end{bmatrix}
  %
  + \begin{bmatrix} F_u(t) \\ F_v(t) \end{bmatrix}.
\end{equation}
Here, $u\in \mathbb{R}^N$ and $v\in \mathbb{R}^N$; $K=K^T$ and $S=-S^T$ are real-valued $N\times N$
matrices.  To solve \eqref{eq_s-v-forcing} numerically, we discretize time on a grid with $t_n = n
\delta_t$, $n=0,1,2,,\ldots$. Here, the time step ($\delta_t$) is constant, but this assumption can
be relaxed. We denote the numerical solution $u^n\approx u(t_n)$ and $v^n\approx v(t_n)$.  In
general, $S(t)\ne0$ and the Hamiltonian system is non-separable. Givent $(u^n, v^n)$, the 
Stromer-Verlet scheme evolves the solution by
\begin{align*}
  \left(I - \frac{\delta_t}{2} S^{n+1/2}\right) \ell_1 &= K^{n+1/2} u^n + S^{n+1/2} v^n + F_v^{n+1/2},\\
%
  v^{n+1/2} &= v^n + \frac{\delta_t}{2}\ell_1,\\
  %
  \kappa_1 &= S^{n} u^n - K^{n} v^{n+1/2} +
  F_u^{n},\\
%
  \left(I - \frac{\delta_t}{2} S^{n+1}\right) \kappa_2 &= S^{n+1}\left( u^n + \frac{\delta_t}{2}
  \kappa_1 \right) - K^{n+1}  v^{n+1/2} + F_u^{n+1},\\
  %
  u^{n+1} &= u^n + \frac{\delta_t}{2}\left( \kappa_1 + \kappa_2 \right),\\
%
  \ell_2 &= K^{n+1/2} u^{n+1} + S^{n+1/2}  v^{n+1/2} + F_v^{n+1/2},\\
  %
  v^{n+1} &= v^n + \frac{\delta_t}{2}\left( \ell_1 + \ell_2 \right).
\end{align*}
In practice, the forcing function may not be available at the half-step. We can then use
\[
F_u^{n+1/2} \approx \frac{1}{2}\left( F_u^{n} + F_u^{n+1}\right).
\]
We note that the scheme becomes explicit when $S(t)=0$. 

The Stromer-Verlet scheme is second order accurate, symplectic and time reversible. It is possible
to write it as a partitioned Runge-Kutta method that combines the trapezoidal and midpoint
rules. The order of accuracy can be raised by using a compositional (Suzuki-Trotter) technique where
each time-step is decomposed into several sub-steps. Fourth order accuracy can be obtained with
three sub-steps, sixth order with seven or nine sub-steps, and eight order requires at least fifteen
sub-steps, see Harirer et al.~\cite{HairerLubichWanner-06} for further details.

\bibliographystyle{plain}
\bibliography{quantum}

\end{document}
