\documentclass{beamer}
\usepackage{amsmath, graphicx, listings, color}

%% \setlength{\topmargin}{0 mm}
%% \setlength{\oddsidemargin}{5 mm}
%% \setlength{\evensidemargin}{5 mm}
%% \setlength{\textwidth}{150 mm}
%% \setlength{\textheight}{210 mm}

%
\newcommand{\calo}{{\cal O}}
%
\newcommand{\ab}{{\bf a}}
\newcommand{\bb}{{\bf b}}
\newcommand{\db}{{\bf d}}
\newcommand{\eb}{{\bf e}}
\newcommand{\fb}{{\bf f}}
\newcommand{\gb}{{\bf g}}
\newcommand{\hb}{{\bf h}}
\newcommand{\ib}{{\bf i}}
\newcommand{\jb}{{\bf j}}
\newcommand{\nb}{{\bf n}}
\newcommand{\pb}{{\bf p}}
\newcommand{\tb}{{\bf t}}
\newcommand{\rb}{{\bf r}}
\newcommand{\xb}{{\bf x}}
\newcommand{\yb}{{\bf y}}
\newcommand{\zb}{{\bf z}}
\newcommand{\qb}{{\bf q}}
\newcommand{\ub}{{\bf u}}
\newcommand{\vb}{{\bf v}}
\newcommand{\wb}{{\bf w}}

\newcommand{\Ab}{{\bf A}}
\newcommand{\Bb}{{\bf B}}
\newcommand{\Eb}{{\bf E}}
\newcommand{\Fb}{{\bf F}}
\newcommand{\Ib}{{\bf I}}
\newcommand{\Hb}{{\bf H}}
\newcommand{\Kb}{{\bf K}}
\newcommand{\Lb}{{\bf L}}
\newcommand{\Pb}{{\bf P}}
\newcommand{\Qb}{{\bf Q}}
\newcommand{\Rb}{{\bf R}}
\newcommand{\Ub}{{\bf U}}
\newcommand{\Tb}{{\bf T}}
\newcommand{\Xb}{{\bf X}}

\newcommand{\uh}{\hat{u}}
\newcommand{\vh}{\hat{v}}
\newcommand{\ph}{\hat{p}}
\newcommand{\qh}{\hat{q}}

\newcommand{\re}{{\rm Re}\,}
\newcommand{\im}{{\rm Im}\,}

\renewcommand{\arraystretch}{1.3}
%
\newcommand{\p}{\partial}
%
\newcommand{\eq}{\!\!\! = \!\!\!}
\newcommand{\om}{\omega}
\newcommand{\divergence}{\nabla\cdot}
\newcommand{\curl}{\nabla\times}

\newcommand{\alphab}{\boldsymbol{\alpha}}
\newcommand{\rhob}{\boldsymbol{\rho}}
\newcommand{\kab}{\boldsymbol{\kappa}}
\newcommand{\lambdab}{\boldsymbol{\lambda}}
\newcommand{\etab}{\boldsymbol{\eta}}
\newcommand{\psib}{\boldsymbol{\psi}}
\newcommand{\Psib}{\boldsymbol{\Psi}}
\newcommand{\zetab}{\boldsymbol{\zeta}}
\newcommand{\sigmab}{\boldsymbol{\sigma}}
\newcommand{\omegab}{\boldsymbol{\omega}}
\newcommand{\Gb}{{\bf G}}
\newcommand{\kb}{{\bf k}}
\newcommand{\sbold}{{\bf s}}

\title{The adjoint method and the trace objective function}
\author{N. Anders Petersson}
\institute{Lawrence Livermore National Laboratory\footnote{LLNL-PRES-abcdef;
This work was performed under the auspices of the U.S. Department of
Energy by Lawrence Livermore National Laboratory under Contract DE-AC52-07NA27344. Lawrence Livermore National Security, LLC.}}
\date{\today}

\begin{document}

\lstset{
  caption=\lstname, captionpos=t, frame=single, basicstyle=\ttfamily\scriptsize,
  columns=fullflexible, keywordstyle=\color{blue},
  commentstyle=\color{red}, stringstyle=\color{gray},
  showspaces=false,
  showstringspaces=false, breaklines=true
}
\renewcommand\lstlistingname{File}
\renewcommand{\thelstlisting}{}
%%%%%%%%%%%%%%%
\frame{\titlepage}

%%%%%%%%%%%%%%%
\begin{frame}{The state equation}
  The state equation for $\Psi_j^\alpha\in {\mathbb C}^{N}$ is
  \[
  \dot{\Psi}_j^\alpha + i H(t,\alpha)\Psi_j^\alpha = 0,\quad 0\leq t\leq T,\quad
  \Psi_j^\alpha(0) = \eb_j,
  \]
  where the Hamiltonian matrix is $H(t,\alphab) = H_0 + p(t,\alphab)H_1 = H^*$.

  The real-valued function $p(t,\alphab)$ depends on the parameters
  \[
  \alphab = [\alpha_0,\alpha_1,\ldots, \alpha_{D-1}]^T.
  \]
  The quantity $\Phi_{jk}^\alpha = \p \Psi_j^\alpha / \p \alpha_k$ satisfies
  \[
  \dot{\Phi}_{jk}^\alpha + iH(t,\alphab)\Phi_{jk}^\alpha = -i\frac{\p H(t,\alphab)}{\p
    \alpha_k} \Psi_j^\alpha,\quad  \Phi_{jk}^\alpha(0) = 0,
  \]
  for $0\leq t\leq T$.
\end{frame}

\begin{frame}{The objective functional}
 Desired target gate after time $T$,
 \[
 \Psib^t(T):=\left[ \db_0, \db_1,\ldots, \db_{D-1}\right]=: G_t,\ G_t^* G_t=I.
 \]
 The control parameters $\alphab$ result in the state
 \[
 \Psib^\alpha(t) = \left[ \Psi^\alpha_0(t), \Psi^\alpha_1(t),\ldots, \Psi^\alpha_{D-1}(t)\right] = G_\alpha(t).
 \]
 The distance between $G_\alpha(t)$ and $G_t$, in weighted trace norm, with a weight function $0\leq w(t) \leq 1$, $w(0)=0$, $w(T)=1$,
 \[
 g_2(\Psib^\alpha) =  \int_0^T w(\tau) \left(
 1 - \frac{1}{D^2} S(\tau)\bar{S}(\tau)\right) \, d\tau,
 \]
where the gate fidelity is measured by projection
 \[
 S(\tau) =  \sum_{j=0}^{D-1}\langle \Psi^\alpha_j(\tau), \db_j\rangle.
 \]
\end{frame}

\begin{frame}{The gradient of the objective functional}
  \begin{multline*}
  \frac{\p g_2}{\p \alpha_k} = -\frac{2}{D^2} {\rm Re\!}\int_0^T w(\tau)
  \frac{\p \bar{S}}{\p \alpha_k}(\tau) S(\tau) \, d\tau \\
  =
  %
  -\frac{2}{D^2} {\rm Re\!}\int_0^T w(\tau) \sum_{j=0}^{D-1}
  %
  \left\langle \db_j,\frac{\Psi^\alpha_j}{\partial \alpha_k}(\tau)\right\rangle S(\tau)\, d\tau\\
  %
  =
  -\frac{2}{D^2} {\rm Re\!} \sum_{j=0}^{D-1}  \int_0^T  w(\tau) \left\langle
  \bar{S}(\tau)\db_j,\frac{\Psi^\alpha_j}{\partial \alpha_k}(\tau)\right\rangle \, d\tau\\
  %
  =
  -\frac{2}{D^2} {\rm Re\!} \sum_{j=0}^{D-1}  \int_0^T  w(\tau) \left\langle
  \bar{S}(\tau)\db_j,\Phi_{jk}^\alpha(\tau)\right\rangle \, d\tau  
  \end{multline*}
  where $\Phi_{jk}^\alpha = \p \Psi_j^\alpha / \p \alpha_k$.
\end{frame}

\begin{frame}{The adjoint relation}
 Let $\Phi = \p \Psi_j^\alpha / \p \alpha_k$. It satisfies the state equation
 \[
 \dot{\Phi} + i H(t,\alphab) \Phi = \fb(t), \quad \Phi(0) = 0.
  \]
  The adjoint equation is
  \[
  - \dot{\lambdab} - i H(t,\alphab) \lambdab = \hb(t),\quad \lambda(T) = 0.
  \]
  Note that the adjoint equation satisfies terminal conditions and is solved backwards in time,
  $T\geq t \geq 0$.

  The solution of the state equation, $\Phi$, with forcing function $\fb$ is related to the solution
  of the adjoint equation, $\lambdab$, with forcing function $\hb$, through the adjoint relation,
  \[
  \int_0^T \langle \fb(\tau), \lambdab(\tau) \rangle\, d\tau =
  \int_0^T \langle \Phi(\tau), \hb(\tau) \rangle\, d\tau.
  \]
\end{frame}


\end{document}

