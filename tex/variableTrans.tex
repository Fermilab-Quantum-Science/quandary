\documentclass[11pt]{article}
\usepackage{amsfonts,amsmath,amssymb,amsthm,graphicx}
\usepackage{multirow}
\usepackage{booktabs}
\usepackage[caption=false]{subfig} 
\usepackage{color}
\usepackage{ifthen}

\setlength{\topmargin}{0 mm}
\setlength{\oddsidemargin}{5 mm}
\setlength{\evensidemargin}{5 mm}
\setlength{\textwidth}{150 mm}
\setlength{\textheight}{210 mm}

\newtheorem{lemma}{Lemma}
\newtheorem{theorem}{Theorem}
\newtheorem{remark}{Remark}
\newcommand{\dzx}{D_0^x}
\newcommand{\dzxp}{D_{0p}^x}
\newcommand{\wdzx}{\widetilde{D_0^x}}
\newcommand{\wdzxp}{\widetilde{D_{0p}^x}}
\newcommand{\dpx}{D_+^x}
\newcommand{\dmx}{D_-^x}
\newcommand{\dzy}{D_0^y}
\newcommand{\dzyp}{D_{0p}^y}
\newcommand{\wdzy}{\widetilde{D_0^y}}
\newcommand{\dpy}{D_+^y}
\newcommand{\dmy}{D_-^y}
\newcommand{\dzz}{D_0^z}
\newcommand{\wdzz}{\widetilde{D_0^z}}
\newcommand{\dpz}{D_+^z}
\newcommand{\dmz}{D_-^z}
\newcommand{\dzt}{D_0^t}
\newcommand{\dpt}{D_+^t}
\newcommand{\dmt}{D_-^t}
\newcommand{\ehx}{E_{1/2}^x}
\newcommand{\ehy}{E_{1/2}^y}
\newcommand{\ehz}{E_{1/2}^z}
%
\newcommand{\calo}{{\cal O}}
%
\newcommand{\ab}{{\mathbf a}}
\newcommand{\bb}{{\mathbf b}}
\newcommand{\db}{{\mathbf d}}
\newcommand{\eb}{{\mathbf e}}
\newcommand{\fb}{{\mathbf f}}
\newcommand{\gb}{{\mathbf g}}
\newcommand{\ib}{{\mathbf i}}
\newcommand{\jb}{{\mathbf j}}
\newcommand{\nb}{{\mathbf n}}
\newcommand{\pb}{{\mathbf p}}
\newcommand{\tb}{{\mathbf t}}
\newcommand{\rb}{{\mathbf r}}
\newcommand{\yb}{{\mathbf y}}
\newcommand{\zb}{{\mathbf z}}
\newcommand{\qb}{{\mathbf q}}
\newcommand{\ub}{{\mathbf u}}
\newcommand{\vb}{{\mathbf v}}
\newcommand{\wb}{{\mathbf w}}

\newcommand{\Ab}{{\mathbf A}}
\newcommand{\Bb}{{\mathbf B}}
\newcommand{\Eb}{{\mathbf E}}
\newcommand{\Fb}{{\mathbf F}}
\newcommand{\Ib}{{\mathbf I}}
\newcommand{\Hb}{{\mathbf H}}
\newcommand{\Kb}{{\mathbf K}}
\newcommand{\Lb}{{\mathbf L}}
\newcommand{\Pb}{{\mathbf P}}
\newcommand{\Qb}{{\mathbf Q}}
\newcommand{\Rb}{{\mathbf R}}
\newcommand{\Ub}{{\mathbf U}}
\newcommand{\Tb}{{\mathbf T}}
\newcommand{\Xb}{{\mathbf X}}

\newcommand{\Abb}{\mathbb{A}}
\newcommand{\Bbbb}{\mathbb{B}}
\newcommand{\Ebb}{\mathbb{E}}
\newcommand{\Fbb}{\mathbb{F}}
\newcommand{\Ibb}{\mathbb{I}}
\newcommand{\Hbb}{\mathbb{H}}
\newcommand{\Kbb}{\mathbb{K}}
\newcommand{\Lbb}{\mathbb{L}}
\newcommand{\Pbb}{\mathbb{P}}
\newcommand{\Qbb}{\mathbb{Q}}
\newcommand{\Rbb}{\mathbb{R}}
\newcommand{\Ubb}{\mathbb{U}}
\newcommand{\Tbb}{\mathbb{T}}
\newcommand{\Xbb}{\mathbb{X}}
\newcommand{\Ybb}{\mathbb{Y}}
\newcommand{\Zbb}{\mathbb{Z}}

\newcommand{\uh}{\hat{u}}
\newcommand{\vh}{\hat{v}}
\newcommand{\ph}{\hat{p}}
\newcommand{\qh}{\hat{q}}

\newcommand{\re}{{\rm Re}\,}
\newcommand{\im}{{\rm Im}\,}

\renewcommand{\arraystretch}{1.3}
%
\newcommand{\p}{\partial}
%
\newcommand{\eq}{\!\!\! = \!\!\!}
\newcommand{\om}{\omega}
%\newcommand{\divergence}{\nabla\cdot}
%\newcommand{\curl}{\nabla\times}

\newcommand{\alphab}{\boldsymbol{\alpha}}
\newcommand{\lambdab}{\boldsymbol{\lambda}}
\newcommand{\psib}{\boldsymbol{\psi}}
\newcommand{\phib}{\boldsymbol{\phi}}
\newcommand{\Psib}{\boldsymbol{\Psi}}

\newcommand{\rhob}{\boldsymbol{\rho}}
\newcommand{\kab}{\boldsymbol{\kappa}}
\newcommand{\etab}{\boldsymbol{\eta}}
\newcommand{\zetab}{\boldsymbol{\zeta}}
\newcommand{\sigmab}{\boldsymbol{\sigma}}
\newcommand{\omegab}{\boldsymbol{\omega}}
\newcommand{\Gb}{{\mathbf G}}
\newcommand{\kb}{{\mathbf k}}
\newcommand{\sbold}{{\mathbf s}}
\newcommand{\ba}{\begin{array}}
\newcommand{\ea}{\end{array}}
\newcommand{\be}{\begin{equation}}
\newcommand{\ee}{\end{equation}}
\newcommand{\bd}{\begin{displaymath}}
\newcommand{\ed}{\end{displaymath}}
\newcommand{\pa}{\partial}
\newcommand{\f}{\frac}
\newcommand{\drp}{D^r_+}
\newcommand{\drm}{D^r_-}
\newcommand{\dqp}{D^q_+}
\newcommand{\dqm}{D^q_-}
\newcommand{\dtqn}{\widetilde{{D^q_0}} }
\newcommand{\dtrn}{\widetilde{{D^r_0}} }
\newcommand{\dqn}{D^q_0}
\newcommand{\drn}{D^r_0}
\newcommand{\erh}{E^r_{1/2}}
\newcommand{\eqh}{E^q_{1/2}}

\def\dpl{D_+}
\def\dmi{D_-}

\newcommand{\ubbar}{\bar{\mathbf{u}}}
\newcommand{\ubar}{\bar{u}}

% Numerical solutions
% \newcommand{\vb}{\mathbf{v}}


% Grids
\newcommand{\xb}{\mathbf{x}}
\newcommand{\ybh}{\hat{\mathbf{x}}}
\newcommand{\xbh}{\hat{\mathbf{x}}}
\newcommand{\Ja}{J_{\alpha}}
\newcommand{\ga}{g_{\alpha}}
\newcommand{\Ma}{M_{\alpha}}

% Interpolation
\newcommand{\Nxy}{N_{\mathbf{x}\rightarrow\hat{\mathbf{x}}}}
\newcommand{\Nyx}{N_{\hat{\mathbf{x}}\rightarrow\mathbf{x}}}
\newcommand{\Nuv}{N_{\bar{\ub}_1\rightarrow\bar{\ub}_2}}
\newcommand{\Nvu}{N_{\bar{\ub}_2\rightarrow\bar{\ub}_1}}
\newcommand{\Nij}{N_{\bar{\ub}_i\rightarrow\bar{\ub}_j}}
\newcommand{\Nji}{N_{\bar{\ub}_j\rightarrow\bar{\ub}_i}}
\newcommand{\Px}{P}
\newcommand{\Pxh}{\hat{P}}

% Domains
\newcommand{\domp}{\Omega_{p}}
\newcommand{\domu}{\Omega_{\ub}}
\newcommand{\domv}{\Omega_{\vb}}
\newcommand{\domui}{\Omega_{\bar{\mathbf{u}}_i}}
\newcommand{\domuj}{\Omega_{\bar{\mathbf{u}}_j}}
\newcommand{\gamja}{\Gamma_{j0}}
\newcommand{\gamjb}{\Gamma_{j1}}
\newcommand{\gamia}{\Gamma_{i0}}
\newcommand{\gamib}{\Gamma_{i1}}

% Comments
\newcommand{\red}{\color{red} AP:}
\newcommand{\usecomments}{true}
\newcommand{\ocomment}[1] {
\ifthenelse{ \equal{\usecomments}{true} }{
 \textbf{Ossian: }{\color{blue} #1}
}{
}
}



\begin{document}

%% \title{Notes on Schroedinger's equation}

%% \author{N. Anders Petersson\thanks{Center for Applied
%%     Scientific Computing, Lawrence Livermore National Laboratory, L-561, PO Box 808, Livermore CA
%%     94551. }}

%% \date{\today}

%% \maketitle
\section{Change of variables}
Consider the scaled Schr\"odinger equation
\begin{equation}\label{eq_schrodinger}
\dot{\psi} = -i H(t) \psi,\quad \psi(0) = \psi_0,\quad H(t) = H_s + f(t)a + \bar{f}(t)a^\dag.
\end{equation}
Here, $f(t)$ is a scalar complex-valued control function. $H_s = H_s^\dag$ is the system Hamiltonian matrix, which we assume to be real, Hermitian and
independent of time. The lowering and raising matrices are denoted $a$ and $a^\dag$,
respectively. These real matrices satisfy
\begin{equation}\label{eq_matrices}
%
a = \begin{bmatrix}
0 & 1 & & & &\\
 & 0 & \sqrt{2} & & &\\
&  & 0 & \sqrt{3} & &\\
& &  & 0 & \sqrt{4} & \\
& &  &  & \ddots & \ddots\\
\end{bmatrix},\quad
%
a^\dag = \begin{bmatrix}
0 &  & & &\\
1 & 0 & & &\\
&  \sqrt{2} & 0 &  &\\
& &  \sqrt{3} & 0 & \\
& &  & \ddots & \ddots
\end{bmatrix}
\end{equation}
We can decompose the control function into its real and imaginary parts,
\[
f(t) = f_r(t) + i f_i(t),
\]
and write the Hamiltonian matrix in \eqref{eq_schrodinger} as
\[
H(t) = H_s + f_r(t)\left( a+ a^\dag\right) + i f_i(t)\left( a - a^\dag\right) =: H_s + K(t) + i S(t).
\]
Here, $K$ and $S$ are real-valued matrices with $K^\dag = K$ and $S^\dag = -S$.

\subsection{Time-independent change of variables}
Because $H_s$ is Hermitian, it has real eigenvalues and can always be diagonalized by a unitary
matrix $V$. By the change of variables, $u(t)=V \psi(t)$, where $V$ is independent of time, Schr\"odinger's equation
\eqref{eq_schrodinger} becomes
\[
\dot{u} = -i VH(t)V^\dag u,\quad VH(t)V^\dag = H_0 + f(t)\tilde{a} + \bar{f}(t)\tilde{a}^\dag,
\]
where $\tilde{a}=V a V^\dag$ and $H_0$ is a real diagonal matrix.

\subsection{Time-dependent unitary transformations}
Consider the unitary tranformation
\[
\psi(t) = U^{\dag}(t)v(t),\quad U^\dag U = I.
\]
We have
\begin{align*}
\dot{\psi} &= \dot{U}^\dag v + U^\dag \dot{v},\\
H\psi &= H U^\dag v
\end{align*}
Thus, \eqref{eq_schrodinger} gives
\[
\dot{U}^\dag v + U^\dag \dot{v} = -i H U^\dag v,
\]
By using the identity $U \dot{U}^\dag = - \dot{U} U^\dag$ and reorganizing the terms,
\[
\dot{v} = -i U H U^\dag v + \dot{U} U^\dag = -i\left( UHU^\dag + i \dot{U} U^\dag \right) v.
\]
Thus, the transformed problem becomes
\begin{equation}\label{eq_timedep_trans}
\dot{v} = -i \tilde{H}(t) v,\quad \tilde{H} = UHU^\dag + i \dot{U} U^\dag.
\end{equation}

\subsection{The interaction picture}

We can construct a unitary transformation based on the time-independent system Hamiltonian $H_s$ in
\eqref{eq_matrices},
\[
U(t) = \exp(i H_s t).
\]
We have $\dot{U}(t) = i H_s \exp(i H_s t)$, which gives
\[
\dot{U}U^\dag =  i H_s \exp(i H_s t) \exp(- i H_s t) = i H_s.
\]
From the definition of the matrix exponential,
\[
\exp(i H_s t) = I + it H_s + \frac{1}{2} (it)^2 H_s^2 + \frac{1}{6} (it)^3 H_s^3 + \ldots.
\]
Thus it is clear that $U=\exp(i H_s t)$ commutes with $H_s$ and $UH_sU^\dag = H_s$. As a result, the
constant (time-independent) part of the transformed Hamiltonian cancels. From
\eqref{eq_timedep_trans}, the transformed problem becomes
\begin{equation}\label{eq_interaction}
  \dot{v} = -i \tilde{H}(t) v,\quad  \tilde{H}(t) = f(t) UaU^\dag + \bar{f}(t) Ua^\dag U^\dag.
\end{equation}
This formulation of Schr\"odinger's equation is known as the interaction picture.

When $H_s=H_0$ is real and diagonal,
\[
H_0=\begin{bmatrix}
\omega_1 & & &\\
& \omega_2 &&\\
& & \omega_3 &\\
& & & \ddots
\end{bmatrix},\quad
%
U(t) = \exp(i H_0 t)=
\begin{bmatrix}
e^{i\omega_1 t} & & &\\
& e^{i\omega_2 t} &&\\
& & e^{i\omega_3 t} &\\
& & & \ddots
\end{bmatrix}
\]
The lowering operator transforms according to
\begin{multline*}
UaU^\dag =
\begin{bmatrix}
  e^{i\omega_1 t} & & & \\
  & e^{i\omega_2 t} & & \\
  & &  e^{i\omega_3 t} & \\
  & & & \ddots
\end{bmatrix}
\begin{bmatrix}
0 & 1 & & &\\
 & 0 & \sqrt{2} & &\\
&  & 0 & \sqrt{3} &\\
& &  & \ddots & \ddots
\end{bmatrix}
\begin{bmatrix}
  e^{-i\omega_1 t} & & & \\
  & e^{-i\omega_2 t} & & \\
  & &  e^{-i\omega_3 t} & \\
  & & & \ddots
\end{bmatrix} = \\
%
\begin{bmatrix}
0 & e^{-i\Delta\omega_1 t} & & &\\
 & 0 & \sqrt{2} e^{-i\Delta\omega_2 t}& &\\
&  & 0 & \sqrt{3} e^{-i\Delta\omega_3 t}  &\\
& &  & \ddots & \ddots
\end{bmatrix} =: D(t)a,
%
\end{multline*}
where $\Delta\omega_k = \omega_{k+1} - \omega_k$ and $D(t)$ is the diagonal matrix
\[
D(t) = \begin{bmatrix}
 e^{-i\Delta\omega_1 t} & & &\\
  & \sqrt{2} e^{-i\Delta\omega_2 t}& &\\
&  & \sqrt{3} e^{-i\Delta\omega_3 t}  &\\
&  &   & \ddots
\end{bmatrix},\quad \Delta\omega_N = 0.
\]
Then,
\[
UaU^\dag = D a,\quad Ua^\dag U^\dag = a^\dag D^\dag.
\]
Note that the value of $\Delta\omega_N$ is arbitrary because it is not needed to evaluate $D a$.
Thus, when the system Hamiltonian is diagonal, the transformed problem corresponding to \eqref{eq_interaction} becomes 
\begin{equation}
  \dot{v} = -i\widetilde{H}(t) v,\quad \widetilde{H}(t) = f(t) D(t) a + \bar{f}(t) a^\dag D^\dag(t).
\end{equation}

The structure of the transformed Hamiltonian indicates how resonance can be induced in the
system. For example, by taking $f(t)=\Omega e^{i\Delta \omega_1 t}$, the Hamiltonian matrix
becomes
\[
\tilde{H}(t) = \begin{bmatrix}
0 & \Omega & & &\\
\bar{\Omega} & 0 & f(t) \sqrt{2} e^{-i\Delta \omega_2 t}& &\\
& \bar{f}(t) \sqrt{2} e^{i\Delta \omega_2 t} & 0 & f(t) \sqrt{3} e^{-i\Delta \omega_3 t}
&\\
\\
& & \ddots & \ddots & \ddots
\end{bmatrix}.
\]
In this case, the dynamics of the system will be dominated by an oscillation between the ground
state and the first excited state, with frequency $\Omega$. One shortcoming this approach is that
the function $f(t)=\Omega e^{i\Delta \omega_1 t}$ is complex-valued. To be realized in the laboratory
it must however be real-valued. We can instead use the ansatz
\begin{multline*}
  f(t) = \alpha \cos(\Delta \omega_1 t) + \beta \sin(\Delta \omega_1 t)\\
  = \frac{\alpha}{2} \left( \exp(i \omega_1 t) + \exp(-i \omega_1 t)\right) - \frac{\beta i}{2}
  \left( \exp(i \omega_1 t) - \exp(-i \omega_1 t)\right)\\
  %
  = \exp(i \omega_1 t) \left( \frac{\alpha}{2}  - \frac{\beta i}{2} \right)  +
  %
  \exp(-i \omega_1 t)\left( \frac{\alpha}{2} + \frac{\beta i}{2} \right)
\end{multline*}
We may write the matrix $D(t)$ as
\[
D(t) = \exp(-i\Delta\omega_1 t) \widetilde{D}(t),\quad
%
\widetilde{D}(t) = \begin{bmatrix}
 1 & & &\\
  & \sqrt{2} e^{i(\Delta\omega_1- \Delta\omega_2) t}& &\\
&  & \sqrt{3} e^{i(\Delta\omega_1 - \Delta\omega_3) t}  &\\
&  &   & \ddots
\end{bmatrix}.
\]
Thus,
\[
f(t) D(t) = \left( \frac{\alpha}{2}  - \frac{\beta i}{2} \right) \widetilde{D}(t) + \exp(-2i \omega_1 t)\left( \frac{\alpha}{2} + \frac{\beta i}{2} \right).
\]


\end{document}
